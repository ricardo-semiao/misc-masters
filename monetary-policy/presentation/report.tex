\documentclass[12pt]{article}


% Geometry
\usepackage[a4paper, left=1.5cm, right=2cm, top=2cm, bottom=2.5cm]{geometry}


% Font encoding
\usepackage[T1]{fontenc} % Font encoding
\usepackage{times}

% Quoting
\usepackage{csquotes}

% Math packages
\usepackage{amsmath} % Basic math symbols and environments
\usepackage{amssymb} % Additional math symbols
\usepackage{amsfonts} % Math fonts
\usepackage{mathtools}
\newcommand\numberthis{\addtocounter{equation}{1}\tag{\theequation}}


% Text packages
\usepackage{parskip}
\setlength{\parskip}{1em}

\usepackage{hyperref}
\usepackage[dvipsnames]{xcolor}
\hypersetup{
    colorlinks=true,
    linkcolor=blue,
    citecolor=YellowOrange!95!black
}


% Pictures
\usepackage{graphicx}
\usepackage{float}
\usepackage{tikz}


% Lists
\usepackage{enumitem}
\setlist[itemize]{itemsep = -0.5em, topsep = -0.5em}


% Bibliography
\usepackage[noibid, giveninits, authordate, maxcitenames=3, hyperref=true]{biblatex-chicago}
\DeclareFieldFormat{url}{Available at\addcolon\space\url{#1}} % for URLs

%% Fontsize and formatting of references
\renewcommand*{\bibfont}{\small}
\setlength\bibitemsep{.25\baselineskip}
\addbibresource{references.bib}


% Title and author
\title{
    \textbf{Unconventional Government Debt Purchases As A Supplement To Conventional Monetary Policy}\\
    {\Large Martin Ellison, Andreas Tischbirek}\\
    Replication report
}

\author{Marcelo Alonso and Ricardo Semião --- FGV-EESP}
\date{December 6, 2024}


\begin{document}

\maketitle



\section{Introduction}

This paper investigates the impact of monetary policy on economic stability and growth. The authors develop a dynamic stochastic general equilibrium (DSGE) model to analyze how different monetary policy rules affect macroeconomic variables such as output, inflation, and employment. The model incorporates various frictions, including price and wage stickiness, which are essential for capturing the real-world complexities of monetary policy transmission. The authors calibrate the model using data from a representative economy and conduct simulations to compare the effects of different policy rules, such as inflation targeting and nominal GDP targeting.

The results indicate that monetary policy significantly influences economic stability and growth. The simulations show that inflation targeting tends to stabilize inflation but may lead to higher output volatility. In contrast, nominal GDP targeting provides a better balance between stabilizing inflation and output. The paper concludes that policymakers should consider the trade-offs associated with different monetary policy rules and suggests that a mixed approach, incorporating elements of both inflation and nominal GDP targeting, may be optimal for achieving macroeconomic stability and growth.

At the end, we propose an extension to the work, regarding the RRPs role in the quantitative easing mechanism.



\section{Model}

In this section, the model is presented, with a somewhat greater degree of derivation then what is shown on the original paper.



\subsection{Households}

The representative household has utility from consumption and dis-utility from labor:

\begin{align*}
    U_{0} &=E_{0}\sum_{t=0}^{\infty}\beta^{t}\left(\chi_{t}^{C}\frac{C_{t}^{1\,-\,\delta}}{1-\delta}-\chi_{t}^{L}\frac{L_{t}^{1\,+\,\psi}}{1+\psi}\right)\\
    C_t &\coloneqq \left(\int_0^1C_t(i)^{\frac{\theta_t - 1}{\theta_t}} di\right)^{\frac{\theta_t}{\theta_t - 1}}
\end{align*}

Where:

\begin{itemize}
    \item $C_t(i)$ is each firms' product consumption.
    \item $C_t$ is a CES consumption aggregate, that depends on the time-varying elasticity of substitution between consumption goods, $\theta_t$. It has a long run trend $\theta$.
    \item $\chi_t^C$ and $\chi_t^L$ are exogenous preference shocks.
    \item $\delta$ and $\psi$ are the weighing parameters.
\end{itemize}

The law of movement for $\ln\chi_t^C$, $\ln\chi_t^L$, and $\ln(\theta_t/\theta)$ AR(1) processes:

\begin{align*}
    \ln(\chi_{t}^{C}) &=\rho_{C}\ln(\chi_{t-1}^{C})+\varepsilon_{t}^{C}, &\varepsilon_{t}^{C} \sim N(0, \sigma^2_C) \tag{1}\\
    \ln(\chi_{t}^{L}) &=\rho_{L}\ln(\chi_{t-1}^{L})+\varepsilon_{t}^{L}, &\varepsilon_{t}^{L} \sim N(0, \sigma^2_L) \tag{2}\\
    \ln\left(\frac{\theta_t}{\theta}\right) &= \rho_\theta \ln\left(\frac{\theta_{t-1}}{\theta}\right) + \varepsilon^\theta_t, &\varepsilon_{t}^{\theta} \sim N(0, \sigma^2_\theta) \tag{10}
\end{align*}

The household maximizes expected utility subject to its budget constraint:

\begin{align*}
    &P_tC_t + T_t + P^S_tS_{t,t+1} = S_{t-1,t} + W_tL_t + (1-t_\pi)(P_tY_t - W_tL_t) \tag{3}\\
    &P_t \coloneqq \left(\int_0^1P_t(i)^{\frac{\theta_t - 1}{\theta_t}} di\right)^{\frac{\theta_t}{\theta_t - 1}}
\end{align*}

Where:

\begin{itemize}
    \item $P_t(i)$ is each firms' product price. And $P_t$ their CES aggregate.
    \item $T_t$ is a lump-sum tax paid to the government.
    \item $S_{x,x+1}$ is the savings instrument purchased at time $x$ with maturity in $x+1$, paying $S_{x,x+1}$. It had cost $P^S_x < 1$ per unit to buy at $x$.
    \item $W_t$ is the nominal wage.
    \item $(P_tY_t - W_tL_t)$ is the dividend from firms, that is taxed at rate $t_\pi$.
    \item All prices are in terms of the numeraire good "money".
\end{itemize}

Lets solve the household problem. The lagrangian and its FOCs are:

\begin{align*}
    \mathcal{L} &= E_{0}\sum_{t=0}^{\infty}\beta^{t}\left(\chi_{t}^{C}\frac{C_{t}^{1\,-\,\delta}}{1-\delta}-\chi_{t}^{L}\frac{L_{t}^{1\,+\,\psi}}{1+\psi}\right)\\
    &~~ - \sum_{t = 0}^\infty\lambda_t(P_tC_t + T_t + P^S_tS_{t,t+1} - S_{t-1,t} - W_tL_t - (1-t_\pi)(P_tY_t - W_tL_t))\\
    \frac{\partial\mathcal{L}}{\partial C_t} &= \beta^t\chi_t^C C_t^{-\delta} - \lambda_tP_t = 0 \tag{H1}\\
    \frac{\partial\mathcal{L}}{\partial L_t} &= -\beta^t\chi_t^L L_t^{\psi} + \lambda_tW_t = 0 \tag{H2}\\
    \frac{\partial\mathcal{L}}{\partial S_{t,t+1}} &= \lambda_tP^S_t - \lambda_{t+1} = 0 \tag{H3}
\end{align*}

Where the dividend term was ignored in (H2) per the hypothesis in footnote 4 "The representative household views dividends as lump-sum income and does not internalise the effect of labour supply on profits.".

Now, we can get $\lambda_t$ from (H1) and plug it into (H2):

\begin{align*}
    &\lambda_t = \frac{\beta^t\chi_t^C C_t^{-\delta}}{P_t}\\
    &-\beta^t\chi_t^L L_t^{\psi} + \frac{\beta^t\chi_t^C C_t^{-\delta}}{P_t}W_t = 0\\
    &\frac{\chi_t^C C_t^{-\delta}}{P_t}W_t = \chi_t^L L_t^{\psi}\\
    &\frac{W_t}{P_t} = \frac{\chi_t^L L_t^{\psi}}{\chi_t^C C_t^{-\delta}(-t_\pi)} = \frac{\chi_t^L }{\chi_t^C}L_t^{\psi}C_t^{\delta} \tag{5}
\end{align*}

Ending with equation (5), the intertemporal optimal relation between consumption and labor.

Now, lets plug the expression for $\lambda_t$ into (H3), and rearrange:

\begin{align*}
    &\lambda_t = \frac{\beta^t\chi_t^C C_t^{-\delta}}{P_t}\\
    &\frac{\beta^t\chi_t^C C_t^{-\delta}}{P_t}P^S_t  = \frac{\beta^{t+1}\chi_{t+1}^C C_{t+1}^{-\delta}}{P_{t+1}}\\
    &\frac{P_{t+1}}{P_t} = \frac{\beta^{t+1}}{\beta}\frac{\chi_{t+1}^C}{\chi_t^C} \left(\frac{C_{t+1}}{C_t}\right)^{-\delta}\frac{1}{P_t^S}\\
    &\Pi_{t+1} = \beta\frac{\chi_{t+1}^C}{\chi_t^C} \left(\frac{C_{t+1}}{C_t}\right)^{-\delta}\frac{1}{P_t^S}\\
    & 1 = \beta\left[\frac{\chi_{t+1}^C}{\chi_t^C} \left(\frac{C_{t+1}}{C_t}\right)^{-\delta}\frac{1}{\Pi_{t+1}}\right]\frac{1}{P_t^S} \tag{4}
\end{align*}

Where $\Pi_t \coloneqq \frac{P_t}{P_{t-1}}$. We get the intertemporal consumption-savings Euler equation.



\subsection{Firms}

There is a continuum of monopolistically competitive firms, indexed by $i \in [0,1]$. Each produce using only labor:

\begin{align*}
    Y_t(i) &= A_tL_t(i)^{\frac{1}{\phi}}\\
    \ln(A_t) &= \rho_A\ln(A_{t-1}) + \varepsilon_t^A, ~~ \varepsilon_t^A \sim N(0, \sigma^2_A),~ |\rho_A < 1| \tag{6}
\end{align*}

Where $\phi$ is a parameter. Price adjustment is a-la Calvo, with a fraction $1 - \alpha$ of firms adjusting their price to $P^*_t(i)$ each period. Then, the aggregate price level is: %cite!!

\begin{align*}
    P_t &= \left((1-\alpha)P^*_t(i)^{1-\theta_t} + \alpha P_{t-1}^{1-\theta_t}\right)^\frac{1}{1-\theta_t}\\
\end{align*}

The firms adjust their price in order to maximize the discounted stream of future profits, subject to the demand constrains:

\begin{align*}
    &E_{t}\sum_{T\,=\,t}^{\infty}\alpha^{T\,-\,t}M_{t,T}[P_{t}(i)Y_{T}(i)\,-\,W_{T}L_{T}(i)] \tag{F1}\\
    Y_{t}(i) &= \left(\frac{P_{t}(i)}{P_{t}}\right)^{-\theta_{t}}Y_{t} \tag{F2}\\
    M_{t,T} &\equiv \beta^{T-t}\frac{\chi_{T}^{C}C_{T}^{-\,\delta}P_{t}}{\chi_{t}^{C}C_{t}^{-\,\delta}P_{T}}
\end{align*}

Where $M_{t,T}$ is the stochastic discount factor, derived from (4).

Lets solve the firm problem. We can invert the firms' production function and use (F2) to simplify (F1):

\begin{align*}
    \max_{P_{t}(i)}\ E_{t}\sum_{T=t}^{\infty}\alpha^{T-t}M_{t,T}\left\{P_{t}(i)Y_{T}\left[{\frac{P_{t}(i)}{P_{T}}}\right]^{-\theta_{t}}-W_{T}\left({\frac{Y_{T}}{A_{T}}}\right)^{\phi}\left[{\frac{P_{t}(i)}{P_{T}}}\right]^{-\theta_{t}\phi}\right\}
\end{align*}

Now, lets derive in terms of $P_t(i)$:

\begin{align*}
    \frac{d}{dP_t(i)} P_{t}(i)Y_{T}\left[{\frac{P_{t}(i)}{P_{T}}}\right]^{-\theta_{t}} &= Y_{T}\left[\frac{P_{t}^{*}(i)}{P_{T}}\right]^{-\theta_{t}}-\theta_{t}Y_{T}\left[\frac{P_{t}^{*}(i)}{P_{T}}\right]^{-\theta_{t}}\\
    \frac{d}{dP_t(i)} W_{T}\left({\frac{Y_{T}}{A_{T}}}\right)^{\phi}\left[{\frac{P_{t}(i)}{P_{T}}}\right]^{-\theta_{t}\phi} &= \theta_{t}\phi\frac{W_{T}}{P_{T}}\left(\frac{Y_{T}}{A_{T}}\right)^{\phi}\left[\frac{P_{t}^{*}(i)}{P_{T}}\right]^{-\theta_{t}\phi-1}
\end{align*}

Thus, the FOC is as below. We can use the definition of the stochastic factor and (5) to substitute the real wage and simplify it.

\begin{align*}
    &E_{t}\sum_{T=\,t}^{\infty}\alpha^{T-t}M_{t,T}\left\{Y_{T}\left[\frac{P_{t}^{*}(i)}{P_{T}}\right]^{-\theta_{t}}-\theta_{t}Y_{T}\left[\frac{P_{t}^{*}(i)}{P_{T}}\right]^{-\theta_{t}}+\theta_{t}\phi\frac{W_{T}}{P_{T}}\left(\frac{Y_{T}}{A_{T}}\right)^{\phi}\left[\frac{P_{t}^{*}(i)}{P_{T}}\right]^{-\theta_{t}\phi-1}\right\}=0\\
    &E_{t}\sum_{T=t}^{\infty}(\alpha\beta)^{T\,-\,t}\chi_{T}^{C}C_{T}^{-\,\delta}Y_{T}\left(\frac{P_{T}}{P_{t}}\right)^{\theta_{t}-1} = \left[\frac{P_{t}}{P_{t}^{*}(i)}\right]^{\theta_{t}\phi\,+\,1\,-\,\theta_{t}}E_{t}\sum_{T\,=\,t}^{\infty}(\alpha\beta)^{T\,-\,t}\frac{\theta_{t}\phi}{\theta_{t}-1}\chi_{T}^{L}L_{T}^{\nu}\left(\frac{Y_{T}}{A_{T}}\right)^{\phi}\left(\frac{P_{T}}{P_{t}}\right)^{\theta_{t}\phi}
\end{align*}

Then, we can clear the notation by defining:

\begin{align*}
    F_{t} &\coloneqq E_{t}\sum_{T\,=\,t}^{\infty}\left(\alpha\beta\right)^{T\,-\,t}\chi_{T}^{C}C_{T}^{\,-\,\delta}Y_{T}\left(\frac{P_{T}}{P_{t}}\right)^{\theta_{t}\,-\,1}\\
    K_{t} &\coloneqq E_{t}\sum_{T\,=\,t}^{\infty}\left(\alpha\beta\right)^{T\,-\,t}\!\frac{\theta_{t}\phi}{\theta_{t}-1}\chi_{T}^{L}L_{T}^{\psi}\left(\frac{Y_{T}}{A_{T}}\right)^{\phi}\left(\frac{P_{T}}{P_{t}}\right)^{\theta_{t}\phi}
\end{align*}

Note that they can be written recursively, $F_t \propto F_{t-1}$ and $K_t \propto K_{t-1}$:

\begin{align*}
    F_{t} &\equiv \chi_{t}^{C}C_{t}^{\,-\,\delta} +  E_{t}\alpha\beta Y_{t}\left(\frac{P_{t}}{P_{t}}\right)^{\theta_{t}\,-\,1}F_{t+1}\\
    F_{t} &\equiv \chi_{t}^{C}C_{t}^{\,-\,\delta} +  \alpha\beta E_{t}Y_{t}\Pi_{t+1}^{\theta_{t}\,-\,1}F_{t+1} \tag{8}\\
    K_{t} &\equiv \frac{\theta_{t}\phi}{\theta_{t}-1}\chi_{t}^{L}L_{t}^{\psi}\left(\frac{Y_{t}}{A_{t}}\right)^{\phi} + E_t \alpha\beta \left(\frac{P_{t}}{P_{t}}\right)^{\theta_{t}\phi}K_{t+1}\\
    K{t} &\equiv \frac{\theta_{t}\phi}{\theta_{t}-1}\chi_{t}^{L}L_{t}^{\psi}\left(\frac{Y_{t}}{A_{t}}\right)^{\phi} + \alpha\beta E_t \Pi_{t+1}^{\theta_{t}\phi}K_{t+1} \tag{9}
\end{align*}

With this, we can simplify the FOC result:

\begin{align*}
    \frac{P^*_{t}(i)}{P_{t}} &= \left(\frac{K_t}{F_t}\right)^\frac{1}{\theta_t(1-\phi)+1} \tag{F3}
\end{align*}

Benigno and Woodford (2005) show that the solution in (F3) is also valid for the recursive forms of $F_t$ and $K_t$. %cite !!!

Additionally, we can rewrite $\frac{P^*_{t}(i)}{P_{t}}$ using the aggregate price equation:

\begin{align*}
    P_t &= \left((1-\alpha)P^*_t(i)^{1-\theta_t} + \alpha P_{t-1}^{1-\theta_t}\right)^\frac{1}{1-\theta_t}\\
    P_t^{1-\theta_t} &= (1-\alpha)P^*_t(i)^{1-\theta_t} + \alpha P_{t-1}^{1-\theta_t}\\
    1 &= \frac{(1-\alpha)P^*_t(i)^{1-\theta_t}}{P_t^{1-\theta_t}} + \frac{\alpha P_{t-1}^{1-\theta_t}}{P_t^{1-\theta_t}}\\
    1 &= (1-\alpha)\left(\frac{P^*_t(i)}{P_t}\right)^{1-\theta_t} + \alpha\left(\frac{P_{t-1}}{P_t}\right)^{1-\theta_t}\\
    (1-\alpha)\left(\frac{P^*_t(i)}{P_t}\right)^{1-\theta_t} &= 1 - \alpha\left(\frac{1}{\Pi_t}\right)^{1-\theta_t}\\
    \left(\frac{P^*_t(i)}{P_t}\right)^{1-\theta_t} &= \frac{1}{1-\alpha}\left(1 - \alpha\Pi_t^{\theta_t - 1}\right)\\
    \frac{P^*_t(i)}{P_t} &= \left(\frac{1 - \alpha\Pi_t^{\theta_t-1}}{1-\alpha}\right)^{\frac{1}{1-\theta_t}} \tag{F4}
\end{align*}

Then, the final form of the FOC is:

\begin{align*}
    \left(\frac{1 - \alpha\Pi_t^{\theta_t-1}}{1-\alpha}\right)^{\frac{1}{1-\theta_t}} = \left(\frac{K_t}{F_t}\right)^\frac{1}{\theta_t(1-\phi)+1} \tag{7}
\end{align*}



\subsection{Banks}

Households deposit saving in the banks, and their revenues are returned to households, since the banking market is perfectly competitive.

Banks decide how to invest the savings between short-term and long-term government bonds, taking the heterogeneous preferences for each matutirty of the households into consideration.

In every period $t$, banks:

\begin{itemize}
    \item Collect $S_{t,t+1}$ savings from households, at price $P^S_t$.
    \item Invest in short-term bonds $B_{t,t+1}$ at price $P^B_t$.
    \item Invest in long-term bonds $Q_{t,t+\tau}$ at price $P^Q_t$.
    \begin{itemize}
        \item A unit of this bond yields $1/\tau$ at each period between $t+1$ and $t+\tau$.
        \item Both prices are known in period $t$ but not in advance.
    \end{itemize}
\end{itemize}

This leads to the flow constraint:

\begin{align*}
    P^S_tS_{t,t+1} &= P^B_tB_{t,t+1} + P^Q_tQ_{t,t+\tau} \tag{11}
\end{align*}

Then, they choice of combination of maturities, given the households preferences, is represented in the problem below:

\begin{align*}
    \max_{B_{t,t+1},Q_{t,t+\tau}}V\left({\frac{B_{t,t+1}}{P_{t}}},{\frac{Q_{t,t+\tau}}{P_{t}}}\right), ~~ s.t. (11)
\end{align*}

Instead of assuming a functional form for $V$, the authors use the Generalised Translog (GTL) model for the indirect utility $V^*$ introduced by Pollak and Wales (1980). %cite!!

\begin{align*}
    V^* &\coloneqq V\left(as^{B^*}, as^{Q^*}\right), ~~ as^{k} \coloneqq \frac{k}{P}, ~~ k \in \{B,Q\}\\
    \log(V_{t}^{*}) &= a_{0}+\sum_{k}a_{1}^{k}\,\log\left({\frac{P_{t}^{k}}{P_{t}^{k}s_{t}-P_{t}^{k}g^{k}-P_{t}^{k}g^{k}}}\right)+\\
    &~~~~{\frac{1}{2}}\sum_{k}\sum_{l}a_{l}^{k l}\,\log\left({\frac{P_{t}^{k}}{P_{t}^{k}s_{t}-P_{t}^{k}g^{k}-P_{t}^{l}g^{k}}}\right)\log\left({\frac{P_{t}^{l}}{P_{t}^{k}s_{t}-P_{t}^{k}g^{-k}P_{t}^{l}g^{k}}}\right)
\end{align*}

Where:

\begin{itemize}
    \item All summations are done for $k,l \in \{B,Q\}$.
    \item $s_t \coloneqq \frac{S_{t,t+1}}{P_t}$.
    \item The asset shares are $as^{k} \coloneqq \frac{P^kk}{P^SS}, ~~ k \in \{B,Q\}$.
    \item $a$'s and $g$'s are the parameters.
\end{itemize}

The objective of the bank is to find the optimal shares $as$. We can find them via $V^*$ using the logarithmic form of Roy's identity\footnote{See Barnett and Serletis (2008) for more information.}. Let $h$ be a indirect utility function, $p$ the vector of prices, and $y$ the wealth (in our case, $P^SS$), then:%cite!!!

\begin{align*}
    s(p,y) &= -\frac{\partial\log h(p,y)/\partial\log p}{\partial\log h(p,y)/\partial\log y}
\end{align*}

We can solve for each price $P^k$ separately:

\begin{align*}
    \frac{\partial\log h(p,y)}{\partial\log p^k} &= \frac{\partial\log \log(V^*)}{\partial\log P^k}\\
    \frac{\partial\log h(p,y)}{\partial\log y} &= \frac{\partial\log \log(V^*)}{\partial\log P^SS}
\end{align*}

Then, in our case:

\begin{align*}
    a s_{t}^{k}=\frac{P_{t}^{k}g^{k}}{P_{t}^{S}s_{t}}+\left(1-\frac{P_{t}^{B}g^{B}+P_{t}^{Q}g^{Q}}{P_{t}^{S}s_{t}}\right)\frac{a_{1}^{k}+\sum_{l}a_{2}^{k l}\log\left(\frac{P_{t}^{l}}{P_{t}^{S}s_{t}-P_{t}^{B}g^{B}-P_{t}^{Q}g^{Q}}\right)}{\sum_{l}a_{1}^{l}+\sum_{k}\sum_{l}a_{2}^{k l}\log\left(\frac{P_{t}^{l}}{P_{t}^{S}s_{t}-P_{t}^{B}g^{B}-P_{t}^{Q}g^{Q}}\right)}
\end{align*}

Now, to find a usable formula for the shares, we must impose restriction on the parameters $a$'s and $g$'s.

\begin{itemize}
    \item Symmetry of the GTL model requires that $a^{kl}_2 = a^{lk}_2$.
    \item The authors argue that the demand for both assets should increase linearly with income, i.e. their Engel curves are linear, i.e. their demand curves are quasi-homothetic. This is translated in the parameters as $a^{BB}_2 + a^{QB}_2 = 0$ and $a^{QQ}_2 + a^{QB}_2 = 0$.
    \item Joining both above, we have that $a^{BB}_2 = a^{QQ}_2 = -a^{QB}_2 = -a^{BQ}_2$. This value will be denoted as $a_2$.
    \item Parameters are normalized such that $a^{BB}_1 = a^{QQ}_1 = 1$. $a^{BB}_1$ will be denoted as $a_1$.
\end{itemize}

With these restrictions, we can rewrite the shares formula as below.

\begin{align*}
    a s_{t}^{B} &= {\frac{P_{t}^{B}g^{B}}{P_{t}^{S}s_{t}}}+\left(1-{\frac{P_{t}^{B}g^{B}+P_{t}^{Q}g^{Q}}{P_{t}^{S}s_{t}}}\right)\\
    &~~~~ \times\left[a_{1}+a_{2}\,\log\left({\frac{P_{t}^{B}}{P_{t}^{S}s_{t}-P_{t}^{B}g^{B}-P_{t}^{Q}g^{Q}}}\right)-a_{2}\,\log\left({\frac{P_{t}^{Q}}{P_{t}^{S}s_{t}-P_{t}^{B}g^{B}-P_{t}^{Q}g^{Q}}}\right)\right]\\
    a s_{t}^{Q} &= \frac{P_{t}^{Q}g^{Q}}{P_{t}^{S}s_{t}}+\left(1-\frac{P_{t}^{B}g^{B}+P_{t}^{Q}g^{Q}}{P_{t}^{S}s_{t}}\right)\\
    &~~~~ \times\left[1-a_{1}-a_{2}\,\log\left(\frac{P_{t}^{B}}{P_{t}^{S}s_{t}-P_{t}^{B}g^{B}-P_{t}^{Q}g^{Q}}\right)+a_{2}\,\log\left(\frac{P_{t}^{Q}}{P_{t}^{S}s_{t}-P_{t}^{B}g^{B}-P_{t}^{Q}g^{Q}}\right)\right]
\end{align*}

Which can be rearranged to cleaner expressions. Lets show it for $as_{t}^{B}$:

\begin{align*}
    as_{t}^{B} = \frac{P^B_t}{P^Ss_t}\frac{B_{t,t+1}}{P_{t}} &= {\frac{P_{t}^{B}g^{B}}{P_{t}^{S}s_{t}}}+\frac{P_{t}^{S}s_{t} - P_{t}^{B}g^{B}-P_{t}^{Q}g^{Q}}{P_{t}^{S}s_{t}}\left[a_{1}+a_{2}\,\log\left(\frac{\frac{P_{t}^{B}}{P_{t}^{S}s_{t}-P_{t}^{B}g^{B}-P_{t}^{Q}g^{Q}}}{\frac{P_{t}^{Q}}{P_{t}^{S}s_{t}-P_{t}^{B}g^{B}-P_{t}^{Q}g^{Q}}}\right)\right]\\
    \frac{B_{t,t+1}}{P_{t}} &= g^{B}\!+\!\frac{P_{t}^{S}s_{t}\!-\!P_{t}^{B}g^{B}\!-\!P_{t}^{Q}g^{Q}}{P_{t}^{B}}\!\left[\!a_{1}\!+\!a_{2}\,\log\left(\!\frac{P_{t}^{B}}{P_{t}^{Q}}\right)\!\right] \tag{12}
\end{align*}

Similar operations can be applied to rearrange the formula for $as_{t}^{Q}$:

\begin{align*}
    \frac{Q_{t,t+\tau}}{P_{t}}\!=\!g^{Q}\!+\!\frac{P_{t}^{S}\!s_{t}\!-\!P_{t}^{B}\!g^{B}\!-\!P_{t}^{Q}g^{Q}}{P_{t}^{Q}}\!\left[1\!-\!a_{1}\!-\!a_{2}\,\log\left(\!\frac{P_{t}^{B}}{P_{t}^{Q}}\right)\right] \tag{13}
\end{align*}

The system for $as_{t}^{B}$ and $as_{t}^{Q}$ must be a Marshallian demand, otherwise it is not consistent with the general equilibrium, the firm's maximization problem. So, it must satisfy the four integrability conditions of Marshallian demands: (i) positivity, (ii) adding up, (iii) homogeneity of degree zero in prices and income and (iv) symmetry and negative semi-definiteness of the matrix of substitution effects. The paper states that "(12) and (13) satisfy (ii) and (iii), (i) will be satisfied by an adequate calibration and (iv) remains to be checked post simulation". %check!!!

The banking market is assumed to be perfectly competitive, so the banks' profits are zero, they return all the revenue to the households. A household that deposited $P^S_{t-1}S_{t-1,t}$ receives $S_{t-1,t}$, which is the sum of the short-term bond yield, and all the long-term bonds' yields that matured in that period:

\begin{align*}
    S_{t-1,t} &= B_{t-1,t} + \frac{1}{\tau}\sum_{j=1}^\tau Q_{t-j,t+\tau-j} \tag{14}
\end{align*}

The implicit interest rates, $i_t^B$ and $i_t^Q$ of both bonds are given by :

\begin{align*}
    1 + i_t &= \frac{1}{P^B_t} \tag{15}\\
    P_{t}^{Q} &= \frac{\frac{1}{\tau}}{1+i_{t}^{Q}}+\frac{\frac{1}{\tau}}{\left(1+i_{t}^{Q}\right)^{2}}+\frac{\frac{1}{\tau}}{\left(1+i_{t}^{Q}\right)^{3}}+\cdots+\frac{\frac{1}{\tau}}{\left(1+i_{t}^{Q}\right)^{\tau}}=\frac{1}{\tau}\frac{1}{1+i_{t}^{Q}}\frac{1-\left(\frac{1}{1+i_{t}^{Q}}\right)^{\tau}}{1-\frac{1}{1+i_{t}^{Q}}} \tag{16}\\
\end{align*}



\subsection{Government}

The government is comprised of two authorities, the treasury and the central bank.

The treasury issues short and long-term government debt.

The amount of long-term bonds issued is $\bar Q_{t,t+\tau}$, and is issued following a simple rule of constant real issuance:

\begin{align*}
    \frac{\bar Q_{t,t+\tau}}{P_t} = fY \tag{17}
\end{align*}

Where:

\begin{itemize}
    \item $f > 0$ is a parameter.
    \item $Y$ is the steady state output.
    \item There is no secondary market of bonds, they are bought from the government and held until maturity.
\end{itemize}

The amount of short-term bonds issued is $\bar B_{t-1,t}$. As only the commercial banks buy this bond, $\bar B_{t-1,t} \equiv B_{t-1,t}$, and we won't use the $\bar B_{t-1,t}$ notation henceforth. Short-term bonds are issued based on the central bank's target for the short term interest rate ($i_t$): given $i_t$, $B_{t-1,t}$ is what makes equation (15) hold.

The government has a consumption good that is given exogenously:

\begin{align*}
    G_t &\coloneqq \left(\int_0^1G_t(i)^{\frac{\theta_t - 1}{\theta_t}} di\right)^{\frac{\theta_t}{\theta_t - 1}}\\
    \ln \left(\frac{G_t}{G}\right) &= \rho_G \ln \left(\frac{G_{t-1}}{G}\right) + \varepsilon_t^G, ~~ \varepsilon_t^G \sim N(0, \sigma^2_G)\tag{18}
\end{align*}

Where $G$ is the steady state value of $G_t$. Lump-sum taxes $T_t$ satisfy $T_t = P_tG_t$.

The central bank's rule for the short-term interest rate is:

\begin{align*}
    \frac{1 + i_t}{1 + i} &= \left(\frac{\Pi_t}{\Pi}\right)^{\gamma_\Pi}\left(\frac{Y_t}{Y}\right)^{\gamma_Y} \nu_t \tag{19}\\
    \ln(\nu_t) &= \rho_\nu \ln(\nu_{t-1}) + \varepsilon_t^\nu, ~~ \varepsilon_t^\nu \sim N(0, \sigma^2_\nu) \tag{20}
\end{align*}

The central bank can purchase or sell $Q^{CB}_{t, t+\tau}$ long-term bonds, following a Taylor-type rule:

\begin{align*}
    \frac{\bar Q_{t, t+\tau} - Q^{CB}_{t, t+\tau}}{\bar Q_{t, t+\tau}} &= \left(\frac{\Pi_t}{\Pi}\right)^{\gamma^{QE}_\Pi}\left(\frac{Y_t}{Y}\right)^{\gamma^{QE}_Y}\xi_t \tag{21}\\
    \ln(\xi_t) &= \rho_\nu \ln(\xi_{t-1}) + \varepsilon_t^\xi, ~~ \varepsilon_t^\xi \sim N(0, \sigma^2_\xi) \tag{22}
\end{align*}

Where:

\begin{itemize}
    \item $\gamma_\Pi, \gamma_Y, \gamma^{QE}_\Pi, \gamma^{QE}_Y > 0$ are parameters that guide how strongly the central bank respond to deviations of the steady state of each variable.
    \item $i$ is the steady state value of the short-term interest rate.
\end{itemize}

The central bank is a net buyer of long-term bonds if $Q^{CB}_{t, t+\tau} > 0$ and vice versa. If $\Pi_t$ and $Y_t$ are "small", then the left-hand-side of (21) must also be "small", i.e. $Q^{CB}_{t, t+\tau} > 0$, injecting money into the economy to stimulate it.

Both authorities are subject to a joint budget constraint:

\begin{align*}
    &P_{t}^{B}B_{t,t+1}+P_{t}^{Q}\bar{Q}_{t,t+\tau}+T_{t}+t_{\pi}(P_{t}Y_{t}-W_{t}L_{t})+\pi_{t}^{C B}=P_{t}G_{t}+B_{t-1,t}+{\frac{1}{\tau}}\sum_{j=1}^{\tau}\bar{Q}_{t-j,t+\tau-j}\\
    &\pi^{CB}_t = {\frac{1}{\tau}}\sum_{j=1}^{\tau}Q^{CB}_{t-j,t+\tau-j} - P^Q_tQ^{CB}_{t,t+\tau}
\end{align*}

Where:

\begin{itemize}
    \item $P_{t}^{B}B_{t,t+1}$ and $P_{t}^{Q}\bar{Q}_{t,t+\tau}$ is the money raised by the treasury.
    \item $T_{t}$ is the lump-sum taxes.
    \item $t_{\pi}(P_{t}Y_{t}-W_{t}L_{t})$ is the tax on firms' profits.
    \item $\pi^{CB}_t$ is the central bank profits (or losses) from asset purchases.
    \item $P_{t}G_{t} \equiv T_{t}$ is the cost of government consumption.
    \item And $B_{t-1,t}+{\frac{1}{\tau}}\sum_{j=1}^{r}\bar{Q}_{t-j,t+\tau-j}$ are the payments on government debt.
\end{itemize}

A useful feature of the model is that this constrain is always satisfied via "the combination of perfect competition in the banking sector and goods markets together with lump-sum funding of current government expenditure". The paper proves this in appendix A.3, but it is not necessary to understand and run the model.



\subsection{Market Clearing}

Demand and supply in the market for long-term bonds clears if:

\begin{align*}
    \bar Q_{t,t+\tau} = Q_{t,t+\tau} + Q^{CB}_{t,t+\tau} \tag{23}
\end{align*}

The resource constraint is given below. If it clears, the market for short-term bonds clears.

\begin{align*}
    Y_t &= C_t + G_t \tag{24}
\end{align*}


Demand and supply in the labour market and clears if hours worked are equal to hours demanded:

\begin{align*}
    L_t &= \int^1_0 L_t(i) ~di\\
    L_t &= \int^1_0 \left(\frac{Y_t(i)}{A_t}\right)^\phi ~di\\
    L_t &= \int^1_0 \left(\frac{Y_t}{A_t}\left(\frac{P_t(i)}{P_t}\right)^{-\theta}\right)^\phi ~di\\
    L_t &= \left(\frac{Y_t}{A_t}\right)^\phi\int^1_0 \left(\frac{P_t(i)}{P_t}\right)^{-\theta\phi} ~di = \left(\frac{Y_t}{A_t}\right)^\phi\int^1_0 D_t ~di\\
    Y_t &= A_T \left(\frac{L_T}{D_t}\right)^\frac{1}{\phi} \tag{25}
\end{align*}

Note that $D_t$ is a measure of price dispersion, a inefficiency on the model. In the section of the firms, we didn't derived its law of motion, lets close the model with it.

Lets use, in order: (i) the fact that only $1-\alpha$ firms adjust their price in each period; and (ii) a arbitrary ordering (without loss of generality) of the firms where the first $i \in [0,\alpha]$ are the ones that can't adjust. Then, we can work on the factor $D_T$:

\begin{align*}
    \int_{0}^{1}\left[{\frac{P_{t}(i)}{P_{t}}}\right]^{-\theta_{t}\phi}d i &=(1-\alpha)\left[{\frac{P_{t}^{*}(i)}{P_{t}}}\right]^{-\theta_{t}\phi}+\alpha\int_{0}^{\alpha}\left[{\frac{P_{t-1}(i)}{P_{t}}}\right]^{-\theta_{t}\phi}d i\\
    \int_{0}^{1}\,\left[\frac{P_{t}(i)}{P_{t}}\right]^{-\theta_{t}\phi}\!d i &=(1-\alpha)\biggl[\frac{P_{t}^{*}(i)}{P_{t}}\biggr]^{-\theta_{t}\phi}+\alpha \Pi_{t}^{\theta_{t}\phi}\int_{0}^{\alpha}\left[\frac{P_{t\,-\,1}(i)}{P_{t\,-\,1}}\right]^{-\theta_{t}\phi}\,d i\\
\end{align*}

Now, note that the last integral is the same wether the upper limit is $\alpha$ or $1$, as the distribution of prices is the same among the constrained firms and all firms in $t-1$.
Lets use the above, plus the definition of $D_t$, and equation (F4) to simplify and get the final equation:

\begin{align*}
    \int_{0}^{1}\,\left[\frac{P_{t}(i)}{P_{t}}\right]^{-\theta_{t}\phi}\!d i &=(1-\alpha)\biggl[\frac{P_{t}^{*}(i)}{P_{t}}\biggr]^{-\theta_{t}\phi}+\alpha \Pi_{t}^{\theta_{t}\phi}\int_{0}^{1}\left[\frac{P_{t\,-\,1}(i)}{P_{t\,-\,1}}\right]^{-\theta_{t}\phi}\,d i\\
    D_t &= (1-\alpha)\biggl[\left(\frac{1 - \alpha\Pi_t^{\theta_t-1}}{1-\alpha}\right)^{\frac{1}{1-\theta_t}}\biggr]^{-\theta_{t}\phi} + \alpha \Pi_{t}^{\theta_{t}\phi}D_{t-1}\\
    D_t &= (1-\alpha)\left(\frac{1 - \alpha\Pi_t^{\theta_t-1}}{1-\alpha}\right)^{\frac{\theta_t\phi}{\theta_t-1}} + \alpha \Pi_{t}^{\theta_{t}\phi}D_{t-1} \tag{26}
\end{align*}



\subsection{Calibration}

The calibration is done in line with previous literature, and all the values are described in table \ref{tb:cal}. The main points are described below, the original text of the paper but organized into items.

\begin{itemize}
    \item The calibration of the parameters in the household and firm problems is standard and in line with Gali (2008) and Smets and Wouters (2003, 2007). %cite!!!
    \item The household discount factor is 0.99.
    \item The inverses of the intertemporal substitution elasticities in consumption and labour supply are 2 and 0.5 respectively.
    \item The intratemporal elasticity of substitution between consumption goods equals 6, which implies a steady-state mark-up of 20 per cent.
    \item The production function exhibits decreasing returns to scale.
    \item Price rigidity is calibrated to a relatively high value, however wages are fully flexible in the model.
    \item Steady-state inflation corresponds to an annual inflation rate of close to 2 per cent.
    \item $\tau$ is 20 so the long-term bond has a maturity of 5 years.
    \item The government spending share of GDP is set to 40 per cent. This is at the upper end for the US and at the lower end for most European countries.
    \item Half of firm profits are paid to the households, the other half to the government, reflecting corporate taxes as well as public ownership of goods producing firms.
    \item The calibration of the asset supply and demand equations generates a steady state in bond markets and a response to central bank asset purchases that aligns with empirical estimates.
    \item Steady-state asset prices $P_B$ and $P_Q$ translate into annualised interest rates on 1-period and 20-period bonds of 4.5\% and 5.5\%, respectively, consistent with average interest rates on 3-month and 5-year US Treasury Bills between 1987 and 2007.
    \item The parameter $f$, equal to 0.66, is the only parameterisation of the issuance equation for long-term bonds consistent with these steady-state bond prices and previously calibrated parameters.
    \item This parameterisation results in long-term government debt accounting for 36\% of the government's total steady-state debt obligations.
    \item A realistic response to central bank asset purchases is achieved through calibrating parameters $a_1$, $a_2$, $g_B$, and $g_Q$ in the asset demand equations.
    \begin{itemize}
        \item $a_2 = 0$ is set without loss of generality because the logarithmic term in the asset demand equations is redundant in a first-order approximation.
        \item The values of $a_1$, $g_B$, $g_Q$ are chosen so that the yield on the long-term bond falls by 100 basis points when the central bank makes asset purchases reducing the present discounted value of long-term payment obligations to the private sector by 1.5\%.
        \item Quantitatively, this means the yield on long-term debt falls by 7 basis points after central bank asset purchases that remove \$100 billion of long-term obligations to the private sector.
        \item This is consistent with empirical estimates of a 3-15 basis points fall in long-term yields per \$100 billion of asset purchases, as summarised in Table 1 of Chen et al. (2012).
    \end{itemize}
    \item The calibration of shock processes highlights the significance of shocks to government spending and the elasticity of substitution, creating reasonable volatility in macroeconomic aggregates.
\end{itemize}

\begin{table}[t]
    \small
    \centering
    \caption{Calibration}
    \label{tb:cal}
    \begin{tabular}{lll}
        \hline Parameter & Value & Description \\ \hline\hline
        $\beta$ & 0.99 & Household discount factor \\
        $\delta$ & 2 & Inverse elasticity of intertemporal substitution in consumption \\
        $\psi$ & 0.5 & Inverse Frisch elasticity of labour supply \\
        $\theta$ & 6 & Steady-state value of intratemporal elasticity of substitution \\
        $\phi$ & 1.1 & Inverse of returns to scale in production \\
        $\alpha$ & 0.85 & Degree of price rigidity \\
        $\Pi$ & 1.005 & Steady-state inflation \\
        $\tau$ & 20 & Horizon of long-term bond \\
        $\bar{g}$ & 0.4 & Steady-state ratio of government spending to GDP \\
        $t_{\mathcal{P}}$ & 0.5 & Share of firm profits received by the government \\
        $f$ & 0.66 & Parameter in long-term bond supply rule \\
        $a_1$ & 0.95 & Asset demand \\
        $a_2$ & 0 & Asset demand \\
        $g^B$ & 10.21 & Asset demand (subsistence level of B) \\
        $g^Q$ & 0.59 & Asset demand (subsistence level of Q) \\
        $\rho_\nu$ & 0.1 & Persistence of shock to interest rate rule \\
        $\rho_{\xi}$ & 0.1 & Persistence of shock to asset purchase rule \\
        $\rho_C$ & 0.1 & Persistence of consumption preference shock \\
        $\rho_L$ & 0.7 & Persistence of labour supply preference shock \\
        $\rho_G$ & 0.1 & Persistence of government spending \\
        $\rho_A$ & 0.7 & Persistence of technology shock \\
        $\rho_\theta$ & 0.95 & Persistence of shock to elasticity of substitution \\
        $\sigma_\nu$ & 0.0025 & Standard deviation of shock to interest rate rule \\
        $\sigma_{\xi}$ & 0.0025 & Standard deviation of shock to asset purchase rule \\
        $\sigma_C$ & 0.0025 & Standard deviation of consumption preference shock \\
        $\sigma_L$ & 0.0025 & Standard deviation of labour supply preference shock \\
        $\sigma_G$ & 0.005 & Standard deviation of government spending shock \\
        $\sigma_A$ & 0.01 & Standard deviation of technology shock \\
        $\sigma_\theta$ & 0.06 & Standard deviation of shock to elasticity of substitution \\
        \hline
    \end{tabular}
\end{table}



\section{Replication Results}


\subsection*{Transmission Mechanisms}

To replicate the impulse response functions (IRFs) displayed in Figures 1 and 2, the files \texttt{RunIRFs1.m} and \texttt{ModeloIRFs.mod} were utilized. These files implement the underlying model and execute the necessary simulations to generate the IRFs. 

The logic and structure of the code, including the steps to execute the model, apply the shocks, and generate the IRFs, are thoroughly explained in the \textit{Code Logic} section.


The economic responses to two types of expansionary monetary policy shocks are depicted in Figures 1 and 2:
\begin{itemize}
    \item \textbf{Figure 1:} Response to an expansionary short-term nominal interest rate shock.
    \item \textbf{Figure 2:} Response to an expansionary shock to purchases of the long-term bond.
\end{itemize}

Both analyses use the following calibration:
\[
\gamma_\Pi = 1.01, \quad \gamma_Y = 0.3, \quad \gamma_\Pi^{QE} = 0, \quad \gamma_Y^{QE} = 60
\]
This calibration implies that, in steady state, a 1\% decrease in output leads the central bank to purchase approximately 45\% of newly issued long-term bonds, representing less than 5\% of the total outstanding long-term debt.

\subsection*{Figure 1: Shock to the Short-Term Interest Rate Rule}

\begin{center}
\begin{tikzpicture}[node distance=2cm, auto]
  % Nodes for Figure 1
  \node (start) [draw, rectangle] {Short-term nominal interest rate (\(i\)) decreases};
  \node (bondprice) [below of=start, draw, rectangle, text width=6cm] {Price of short-term bonds (\(P^B\)) increases};
  \node (demandfall) [below of=bondprice, draw, rectangle, text width=6cm] {Demand for short-term bonds falls};
  \node (savingprice) [below of=demandfall, draw, rectangle, text width=6cm] {Price of composite savings (\(P^S s\)) rises, savings decrease};
  \node (outputconsumption) [below of=savingprice, draw, rectangle, text width=6cm] {Output (\(Y\)) and consumption (\(C\)) increase};
  \node (inflation) [below of=outputconsumption, draw, rectangle, text width=6cm] {Inflation (\(\Pi\)) rises above steady-state};

  % Arrows for Figure 1
  \draw[->] (start) -- (bondprice);
  \draw[->] (bondprice) -- (demandfall);
  \draw[->] (demandfall) -- (savingprice);
  \draw[->] (savingprice) -- (outputconsumption);
  \draw[->] (outputconsumption) -- (inflation);
\end{tikzpicture}
\end{center}




\subsection*{Figure 2: Shock to the Long-Term Bond Purchase Rule}

\begin{center}
\begin{tikzpicture}[node distance=2cm, auto]
  % Nodes for Figure 2
  \node (start2) [draw, rectangle, text width=12cm, align=center] {Central bank purchases of long-term bonds (\(q^{CB}\)) increase};
  \node (bondprice2) [below of=start2, draw, rectangle, text width=12cm, align=center] {Supply of long-term bonds to private sector decreases, (\(P^Q\)) increases};
  \node (savingprice2) [below of=bondprice2, draw, rectangle, text width=12cm, align=center] {Price of composite savings (\(P^S s\)) rises, households save less, consume more, and supply more labour};
  \node (flattening) [below of=savingprice2, draw, rectangle, text width=12cm, align=center] {Higher \(P^Q\) leads to a decline in the long-term interest rate (Flattening of the yield curve)};
  \node (outputinflation) [below of=flattening, draw, rectangle, text width=12cm, align=center] {Output (\(Y\)) increases and inflation (\(\Pi\)) rises};

  % Arrows for Figure 2
  \draw[->, line width=1mm] (start2.south) -- (bondprice2.north);
  \draw[->, line width=1mm] (bondprice2.south) -- (savingprice2.north);
  \draw[->, line width=1mm] (savingprice2.south) -- (flattening.north);
  \draw[->, line width=1mm] (flattening.south) -- (outputinflation.north);
\end{tikzpicture}
\end{center}





\vspace{0.2cm}

The variables are presented in the following order: \textbf{(1) inflation ($\Pi$), (2) output ($Y$), (3) short-term nominal interest rate ($i$), (4) price of short-term bonds ($P^B$), (5) price of long-term bonds ($P^Q$), (6) composite savings price ($P^S s$), (7) adjusted debt position ($b$), (8) quantity of long-term bonds ($q$), (9) household deposits ($s$), (10) central bank bond holdings ($q^{CB}$), (11) hours worked ($L$), and (12) real wages ($w$)}.









\begin{figure}[H]
    \centering
    \includegraphics[width=1\linewidth]{Figura1.png}
    \caption{Response to an expansionary short-term nominal interest rate shock}
    \label{fig:enter-label}
\end{figure}





\begin{figure}[H]
    \centering
    \includegraphics[width=1\linewidth]{Figura2.png}
    \caption{Response to an expansionary shock to purchases of the long-term bond}
    \label{fig:enter-label}
\end{figure}












\subsection*{Optimized Policy Rules}


\subsection*{Table 2 - Optimised policy rules with interest rate stabilisation}




\begin{table}[H]
\centering
\caption{Optimised policy rules without interest rate stabilisation.}
\label{tab:policy_rules_no_stabilisation}
\resizebox{\textwidth}{!}{%
\begin{tabular}{ccccccccccc}
\hline
\textbf{Weights (in \%)} & \multicolumn{2}{c}{\textbf{Interest rate rule (\( \gamma_\Pi, \gamma_Y \))}} & \multicolumn{2}{c}{\textbf{Asset purchase rule (\( \gamma_\Pi^{QE}, \gamma_Y^{QE} \))}} & \textbf{Var(\( \Pi \))} & \textbf{Var(\( Y \))} & \textbf{Var(\( i \))} & \textbf{Var(\( i^Q \))} & \textbf{Loss} & \(\Delta r^*\) \textbf{equ. (\%)} \\ 
\( (\omega_\Pi, \omega_Y, \omega_i, \omega_{iQ}) \) & & & & & \(\times 10^{-5}\) & \(\times 10^{-5}\) & \(\times 10^{-4}\) & \(\times 10^{-4}\) & & \\ 
\hline
70, 30, 0, 0 & 1.48, 2.22 & & 0(f), 0(f) & & 9.33 & 1.43 & 1.59 & 0.85 & 6.96 & \\
             & 1.48(f), 2.22(f) & & 0.05, 1.87 & & 9.36 & 1.16 & 1.62 & 0.60 & 6.90 \\
             & 1.66, 0.00 & & 0.00, 18.59 & & 7.71 & 1.35 & 2.16 & 1.16 & 5.80 & 1.64 \\
\hline
80, 20, 0, 0 & 1.49, 2.16 & & 0(f), 0(f) & & 9.30 & 1.53 & 1.58 & 0.88 & 7.75 & \\
             & 1.49(f), 2.16(f) & & 0.04, 1.78 & & 9.29 & 1.23 & 1.61 & 0.62 & 7.68  \\
             & 1.67, 0.00 & & 0.00, 18.22 & & 7.68 & 1.42 & 2.11 & 1.14 & 6.43 & 1.63 \\
\hline
90, 10, 0, 0 & 1.49, 2.11 & & 0(f), 0(f) & & 9.28 & 1.62 & 1.58 & 0.91 & 8.52 & \\
             & 1.49(f), 2.11(f) & & 0.04, 1.70 & & 9.24 & 1.30 & 1.60 & 0.64 & 8.44  \\
             & 1.67, 0.00 & & 0.00, 17.92 & & 7.67 & 1.48 & 2.20 & 1.13 & 7.05 & 1.63 \\
\hline
\end{tabular}%
}
\begin{flushleft}
\textbf{Notes:} \( f \): fixed, i.e., not chosen optimally. \\
\end{flushleft}
\end{table}



As described in the \textbf{Code Logic} section, the \texttt{tabela2.m} script automates the optimization process for various monetary policy configurations using the \texttt{UnconventionalModel\_cases.mod} file. It generates output files, such as \texttt{resultadosTabela2\_Caso 1.1.txt} and \texttt{resultadosTabela2\_Caso 1.2.txt}, containing the optimized parameters, loss values, and variances of macroeconomic variables (\(\Pi\), \(Y\), \(i\), \(i^Q\)). These results match those presented in \textbf{Table 2}.


Using the results from the generated text files, we calculated the final column of \textbf{Table 2} (\(\Delta \pi^*\)), which represents the equivalent annualized change in steady-state inflation required to equate the loss of the alternative policy with that of the baseline policy. This calculation follows the methodology described in Appendix A.5 of the paper.


The loss function quantifies the deviations of key macroeconomic variables from their steady-state values. It is used to evaluate the effectiveness of monetary policy rules.

The loss from the \textbf{baseline} policy is defined as:
\begin{equation}
\Omega^b = \omega_\Pi \, \text{Var}(\Pi_t^b - \Pi^b) + \omega_Y \, \text{Var}(Y_t^b - Y^b) + \omega_i \, \text{Var}(i_t^b - i^b) + \omega_Q \, \text{Var}(i_t^{Q,b} - i^{Q,b}) \tag{A.37}
\end{equation}

Similarly, the loss from an \textbf{alternative} policy is given by:
\begin{equation}
\Omega^a = \omega_\Pi \, \text{Var}(\Pi_t^a - \Pi^a) + \omega_Y \, \text{Var}(Y_t^a - Y^a) + \omega_i \, \text{Var}(i_t^a - i^a) + \omega_Q \, \text{Var}(i_t^{Q,a} - i^{Q,a}) \tag{A.38}
\end{equation}

The variances measure the volatility of inflation ($\Pi$), output ($Y$), short-term interest rates ($i$), and long-term interest rates ($i^Q$). The weights $\omega_\Pi$, $\omega_Y$, $\omega_i$, and $\omega_Q$ reflect the central bank's preferences for stabilizing each variable.

By expanding the inflation variance in the alternative policy, we have:
\begin{equation}
\Omega^a = \omega_\Pi \, \mathbb{E}[(\Pi_t^a - \Pi^a)^2] + \omega_Y \, \text{Var}(Y_t^a - Y^a) + \omega_i \, \text{Var}(i_t^a - i^a) + \omega_Q \, \text{Var}(i_t^{Q,a} - i^{Q,a}) \tag{A.39}
\end{equation}

To compare the alternative policy with the baseline, we define the \textbf{equivalent change in steady-state inflation} $\Delta \pi^*$ as the additional inflation required to equalize losses:
\begin{equation}
\omega_\Pi \, \mathbb{E}[(\Pi_t^a + \Delta \pi^* - \Pi^a)^2] + \omega_Y \, \text{Var}(Y_t^a - Y^a) + \omega_i \, \text{Var}(i_t^a - i^a) + \omega_Q \, \text{Var}(i_t^{Q,a} - i^{Q,a}) = \Omega^b \tag{A.40}
\end{equation}

Rewriting, we solve for $\Delta \pi^*$:
\begin{equation}
\omega_\Pi [\mathbb{E}[(\Pi_t^a - \Pi^a)^2] + (\Delta \pi^*)^2] + \omega_Y \, \text{Var}(Y_t^a - Y^a) + \omega_i \, \text{Var}(i_t^a - i^a) + \omega_Q \, \text{Var}(i_t^{Q,a} - i^{Q,a}) = \Omega^b \tag{A.41}
\end{equation}

\begin{equation}
\Omega^a + \omega_\Pi (\Delta \pi^*)^2 = \Omega^b \tag{A.42}
\end{equation}

\begin{equation}
\Delta \pi^* = \sqrt{\frac{\Omega^b - \Omega^a}{\omega_\Pi}} \tag{A.43}
\end{equation}

Finally, the annualized equivalent change in steady-state inflation is computed as:
\begin{equation*}
\Delta \pi^*_{\text{annual}} = \left(1 + \Delta \pi^*_{\text{quarter}}\right)^4 - 1
\end{equation*}



\textbf{Case Calculations}

\textbf{Case 1:}  
Baseline Loss (\(\Omega^b\)): \(6.96 \times 10^{-5}\)  
Alternative Loss (\(\Omega^a\)): \(5.80 \times 10^{-5}\)  

 
\[
\Delta \pi^*_\text{quarter} = \sqrt{\frac{6.96 - 5.80}{0.7}}
\]

\begin{itemize}
    \item Subtraction: \(6.96 - 5.80 = 1.16\)
    \item Multiplication by \(10^{-5}\): \(1.16 \times 10^{-5} = 0.0000116\)
    \item Division: \(\frac{0.0000116}{0.7} \approx 0.0000165714\)
    \item Square Root: \(\sqrt{0.0000165714} \approx 0.00407\)
\end{itemize}



\textbf{Annualized Result:}  
\[
\Delta \pi^*_\text{annual} = \left(1 + 0.00407\right)^4 - 1 \approx 0.0164 \, \text{or } 1.64\%
\]

---

\textbf{Case 2:}  
Baseline Loss (\(\Omega^b\)): \(7.75 \times 10^{-5}\)  
Alternative Loss (\(\Omega^a\)): \(6.43 \times 10^{-5}\)  

 
\[
\Delta \pi^*_\text{quarter} = \sqrt{\frac{7.75 - 6.43}{0.8}}
\]

\begin{itemize}
    \item Subtraction: \(7.75 - 6.43 = 1.32\)
    \item Multiplication by \(10^{-5}\): \(1.32 \times 10^{-5} = 0.0000132\)
    \item Division: \(\frac{0.0000132}{0.8} = 0.0000165\)
    \item Square Root: \(\sqrt{0.0000164} \approx 0.00405\)
\end{itemize}


\textbf{Annualized Result:}  
\[
\Delta \pi^*_\text{annual} = \left(1 + 0.00406\right)^4 - 1 \approx 0.0163 \, \text{or } 1.63\%
\]

---

\textbf{Case 3:}  
Baseline Loss (\(\Omega^b\)): \(8.52 \times 10^{-5}\)  
Alternative Loss (\(\Omega^a\)): \(7.05 \times 10^{-5}\)  


\[
\Delta \pi^*_\text{quarter} = \sqrt{\frac{8.52 - 7.05}{0.9}}
\]

\begin{itemize}
    \item Subtraction: \(8.52 - 7.05 = 1.47\)
    \item Multiplication by \(10^{-5}\): \(1.47 \times 10^{-5} = 0.0000147\)
    \item Division: \(\frac{0.0000147}{0.9} \approx 0.0000163333\)
    \item Square Root: \(\sqrt{0.0000163333} \approx 0.00404\)
\end{itemize}



\textbf{Annualized Result:}  
\[
\Delta \pi^*_\text{annual} = \left(1 + 0.00404\right)^4 - 1 \approx 0.0163 \, \text{or } 1.63\%
\]











\subsection*{Table 3 - Optimised policy rules with interest rate stabilisation}

As described in the \textbf{Code Logic} section, the \texttt{tabela3.m} script automates the optimization process for different monetary policy configurations using the same \texttt{UnconventionalModel\_cases.mod} file as in Table 2. The script generates output files for each subcase, such as \texttt{resultadosTabela3\_Caso 1.1.txt}, \texttt{resultadosTabela3\_Caso 1.2.txt}, and so on. These files contain key information, including the optimized parameters, loss values, and variances of the macroeconomic variables (\(\Pi\), \(Y\), \(i\), and \(i^Q\)), which correspond to the results presented in \textbf{Table 3}.

The last column of \textbf{Table 3} (\(\Delta \pi^*\)) represents the equivalent annualized change in steady-state inflation required to match the loss of the alternative policy to that of the baseline policy. The calculation follows the same methodology described in Appendix A.5 and previously used for \textbf{Table 2}.





\begin{table}[H]
\centering
\caption{Optimised policy rules with interest rate stabilisation.}
\label{tab:policy_rules_interest_rate_stabilisation}
\resizebox{\textwidth}{!}{%
\begin{tabular}{ccccccccccc}
\hline
\textbf{Weights (in \%)} & \multicolumn{2}{c}{\textbf{Interest rate rule (\( \gamma_\Pi, \gamma_Y \))}} & \multicolumn{2}{c}{\textbf{Asset purchase rule (\( \gamma_\Pi^{QE}, \gamma_Y^{QE} \))}} & \textbf{Var(\( \Pi \))} & \textbf{Var(\( Y \))} & \textbf{Var(\( i \))} & \textbf{Var(\( i^Q \))} & \textbf{Loss} & \(\Delta \pi^*\) \textbf{equ. (\%)} \\ 
\( (\omega_\Pi, \omega_Y, \omega_i, \omega_{iQ}) \) & & & & & \(\times 10^{-5}\) & \(\times 10^{-5}\) & \(\times 10^{-4}\) & \(\times 10^{-4}\) & & \\ 
\hline
70, 20, 10, 0 & 1.55, 2.26 & & 0(f), 0(f) & & 9.31 & 1.58 & 1.55 & 0.91 & 8.38 & \\
              & 1.55(f), 2.26(f) & & 1.78, 4.33 & & 9.50 & 1.60 & 1.18 & 1.07 & 8.15 & \\
              & 1.00, 0.00 & & 4.34, 16.14 & & 8.45 & 1.31 & 0.89 & 1.53 & 7.07 & 1.74 \\
\hline
70, 10, 10, 10 & 1.51, 2.27 & & 0(f), 0(f) & & 9.34 & 1.43 & 1.58 & 0.86 & 9.12 & \\
               & 1.51(f), 2.27(f) & & 0.69, 4.07 & & 9.46 & 1.10 & 1.10 & 0.61 & 8.82 & \\
               & 1.36, 0.40 & & 1.10, 11.66 & & 8.24 & 1.50 & 1.46 & 0.54 & 7.91 & 1.67 \\
\hline
\end{tabular}%
}
\begin{flushleft}
\textbf{Notes:} \( f \): fixed, i.e., not chosen optimally. \\
\end{flushleft}
\end{table}






Table 3 highlights the role of unconventional monetary policy in reducing losses when the central bank considers interest rate volatility in its decision-making process. Unlike Table 2, where the focus is solely on inflation and output stabilization, Table 3 incorporates weights on both short-term (\(i\)) and long-term (\(i^Q\)) interest rate variances in the loss function.

The results suggest that unconventional monetary policy becomes significantly more effective when the central bank is concerned about the volatility of interest rates. When only short-term interest rate variability is costly (\(\omega_{iQ} = 0\)), the optimal policy heavily relies on long-term bond purchases. In these cases, unconventional policy nearly replaces conventional short-term interest rate policies. This occurs because long-term interest rate volatility caused by bond purchases is not penalized in the loss function, allowing the central bank to reduce the volatility of short-term rates without incurring additional costs. Consequently, the optimal parameters for the short-term interest rate rule (\(gampi\) and \(gamY\)) are often at their lower bounds, reflecting minimal reliance on conventional policies.

As \(\omega_{iQ}\) becomes positive, the gains from unconventional policies remain substantial but are slightly diminished. This is because the central bank now partially accounts for the volatility introduced in long-term interest rates by bond purchases. Despite this, Table 3 shows that the combined use of conventional and unconventional tools still leads to larger welfare gains compared to Table 2, where interest rate volatility is not considered. 

The last column of Table 3 (\(\Delta \pi^*\)) provides the equivalent steady-state inflation increase that would produce the same loss level as the alternative policies compared to the baseline. The calculation of \(\Delta \pi^*\) is analogous to the methodology used for Table 2, reflecting the relative effectiveness of unconventional tools in different scenarios. For example, when both \(gampiQE\) and \(gamYQE\) are optimized, the resulting gains, as reflected by the lower \(\Delta \pi^*\), underscore the effectiveness of incorporating long-term bond purchases into the central bank's toolkit.











\subsection*{Table 4 - Losses by type of shock}



\begin{table}[ht]
\centering
\caption{Losses by Type of Shock}
\label{tab:simulation_results}
\begin{tabular}{ccccccccc}
\hline
\textbf{Shock} & \textbf{Interest rate} & \textbf{Asset purchase} & \textbf{Var(\(\pi_t\))} & \textbf{Var(\(Y_t\))} & \textbf{Var(\(i_t\))} & \textbf{Var(\(i_t^Q\))} & \textbf{Loss} \\
& \textbf{rule (\(\gamma_{\pi}, \gamma_{Y}\))} & \textbf{rule (\(\gamma_{\pi}^{QE}, \gamma_{Y}^{QE}\))} & $\times 10^{-5}$ & $\times 10^{-5}$ & $\times 10^{-4}$ & $\times 10^{-4}$ & \\
\hline
$\nu$ & 1.49, 2.16 & 0, 0 & 0.5106 & 0.0586 & 0.0640 & 0.0495 & 0.4202 \\
$\nu$ & 1.49, 2.16 & 0.04, 1.78 & 0.4835 & 0.0481 & 0.0662 & 0.0380 & 0.3964 \\
$\nu$ & 1.67, 0.00 & 0.00, 18.22 & 0.4293 & 0.0624 & 0.1447 & 0.0468 & 0.3559 \\
$\xi$ & 1.49, 2.16 & 0, 0 & 0.0059 & 0.0004 & 0.0010 & 0.0055 & 0.0048 \\
$\xi$ & 1.49, 2.16 & 0.04, 1.78 & 0.0056 & 0.0004 & 0.0010 & 0.0052 & 0.0045 \\
$\xi$ & 1.67, 0.00 & 0.00, 18.22 & 0.0052 & 0.0005 & 0.0015 & 0.0032 & 0.0042 \\
$\epsilon^C$ & 1.49, 2.16 & 0, 0 & 0.2177 & 0.0775 & 0.0273 & 0.0211 & 0.1897 \\
$\epsilon^C$ & 1.49, 2.16 & 0.04, 1.78 & 0.2440 & 0.0711 & 0.0317 & 0.0177 & 0.2094 \\
$\epsilon^C$ & 1.67, 0.00 & 0.00, 18.22 & 0.1565 & 0.0743 & 0.0443 & 0.0949 & 0.1401 \\
$\epsilon^L$ & 1.49, 2.16 & 0, 0 & 0.0153 & 0.0016 & 0.0016 & 0.0016 & 0.0126 \\
$\epsilon^L$ & 1.49, 2.16 & 0.04, 1.78 & 0.0148 & 0.0012 & 0.0017 & 0.0012 & 0.0121 \\
$\epsilon^L$ & 1.67, 0.00 & 0.00, 18.22 & 0.0143 & 0.0017 & 0.0040 & 0.0005 & 0.0117 \\
$\epsilon^G$ & 1.49, 2.16 & 0, 0 & 1.5481 & 0.5512 & 0.1940 & 0.1504 & 1.3487 \\
$\epsilon^G$ & 1.49, 2.16 & 0.04, 1.78 & 1.7350 & 0.5058 & 0.2252 & 0.1256 & 1.4891 \\
$\epsilon^G$ & 1.67, 0.00 & 0.00, 18.22 & 1.1132 & 0.5283 & 0.3150 & 0.6752 & 0.9962 \\
$\epsilon^A$ & 1.49, 2.16 & 0, 0 & 0.6660 & 0.0712 & 0.0691 & 0.0713 & 0.5470 \\
$\epsilon^A$ & 1.49, 2.16 & 0.04, 1.78 & 0.6442 & 0.0539 & 0.0745 & 0.0531 & 0.5261 \\
$\epsilon^A$ & 1.67, 0.00 & 0.00, 18.22 & 0.6209 & 0.0746 & 0.1757 & 0.0226 & 0.5117 \\
$\theta$ & 1.49, 2.16 & 0, 0 & 6.3539 & 0.8062 & 1.2220 & 0.5755 & 5.2443 \\
$\theta$ & 1.49, 2.16 & 0.04, 1.78 & 6.1758 & 0.5705 & 1.2095 & 0.3656 & 5.0547 \\
$\theta$ & 1.67, 0.00 & 0.00, 18.22 & 5.4324 & 0.6304 & 1.5373 & 0.2897 & 4.4720 \\
\hline
\end{tabular}
\end{table}





To assess the impact of different policy regimes on losses attributable to each type of shock, the author considers the case where the weights in the loss function are set to $(\omega_{\Pi}, \omega_{Y}, \omega_{i}, \omega_{Q}) = (80, 20, 0, 0)$. In Table 4, the variances of all shocks except one are set to zero. The shock of interest is indicated in the first column and has its variance calibrated as in Table 1. Policy parameters, when not set to zero, take the value that is optimal when all shocks are operating jointly.

\textbf{We explain the methodology used to construct Table 4 in the \textit{Code Logic} section.}
















\subsection*{Figure 3 - Response to a shock to the elasticity of substitution}


Figure 3 illustrates the dynamics of two monetary policy approaches in response to a shock to the elasticity of substitution: the baseline scenario (black lines) and the combined policy scenario (blue lines). In the baseline, the central bank relies exclusively on short-term interest rate adjustments to respond to deviations in both inflation and output. In contrast, the combined policy approach uses short-term rates primarily to stabilize inflation, while quantitative easing (QE) is employed to address output fluctuations.


The results highlight the effectiveness of separating stabilization tools. While the baseline policy struggles with the dual burden on short-term rates, the combined policy leverages QE to achieve smoother output dynamics and more stable inflation, ensuring a more balanced response to economic shocks.

\textbf{We explain the methodology used to construct Figure 3 in the \textit{Code Logic} section.}





\begin{figure}[H]
    \centering
    \includegraphics[width=1\linewidth]{Figura3.png}
    \caption{Response to a shock to the elasticity of substitution for short-term interest policy only (Black) and with conventional and unconventional policy
(Blue).}
    \label{fig:enter-label}
\end{figure}
































\appendix



\section{Stationary Model}

Equations (1) to (26) describe the economy in a rational expectations equilibrium. But for $\Pi > 0$, nominal variables have a tendency to grow indefinitely, thus, a transformation will be needed.

The objective is to remove nominal trends and express all quantities in terms of real values or ratios that do not grow indefinitely in the presence of inflation. For this purpose, nominal variables are divided by \(P_t\) (the price level at time \(t\)), and inflation is introduced as \(\Pi_t \equiv \frac{P_t}{P_{t-1}}\).

The main transformations are:

\begin{itemize}
    \item Real wage: \(w_t \equiv \frac{W_t}{P_t}\),
    \item Real savings: \(s_t \equiv \frac{S_{t,t+1}}{P_t}\),
    \item Real short-term bonds: \(b_t \equiv \frac{B_{t,t+1}}{P_t}\),
    \item Real long-term bonds: \(q_t \equiv \frac{Q_{t,t+\tau}}{P_t}\),
    \item Issuance of long-term bonds in real terms: \(\bar{q}_t \equiv \frac{\bar{Q}_{t,t+\tau}}{P_t}\),
    \item Long-term bonds held by the Central Bank in real terms: \(q_t^{CB} \equiv \frac{Q_{t,t+\tau}^{CB}}{P_t}\).
\end{itemize}

These transformations are applied to all relevant equations. We will follow the order presented appendix B of the original paper, where each equation is numbered as (B.x).


\subsection{Equation (B.1): Household Budget Constraint}

The nominal budget constraint is (equation (3) of the original text):

\[
P_t C_t + T_t + P_t^S S_{t,t+1} = S_{t-1,t} + W_t L_t + (1 - t_\pi)(P_t Y_t - W_t L_t).
\]

Dividing the entire equation by \(P_t\):

\[
C_t + \frac{T_t}{P_t} + \frac{P_t^S S_{t,t+1}}{P_t} = \frac{S_{t-1,t}}{P_t} + \frac{W_t}{P_t} L_t + (1 - t_\pi) \left(Y_t - \frac{W_t}{P_t} L_t\right).
\]

Note that \(\frac{T_t}{P_t} = G_t\), since \(T_t = P_t G_t\). Also, we define \(\frac{W_t}{P_t} = w_t\) and \(\frac{S_{t,t+1}}{P_t} = s_t\). For \(\frac{S_{t-1,t}}{P_t}\), since \(S_{t-1,t}\) is chosen in \(t-1\) but expressed in nominal terms of period \(t\), we rewrite:

\[
\frac{S_{t-1,t}}{P_t} = \frac{S_{t-1,t}}{P_{t-1}} \cdot \frac{P_{t-1}}{P_t} = \frac{s_{t-1}}{\Pi_t}.
\]

Here, \(\Pi_t = \frac{P_t}{P_{t-1}}\) represents inflation. To express \(S_{t-1,t}\) in real terms of period \(t\), we discount it by inflation between \(t-1\) and \(t\).

Substituting all transformations:

\[
C_t + G_t + P_t^S s_t = \frac{s_{t-1}}{\Pi_t} + w_t L_t + (1 - t_\pi)(Y_t - w_t L_t) + t_\pi w_t L_t.
\]

Where did the separate term \(t_\pi w_t L_t\) come from? Originally, we had \((1 - t_\pi)(Y_t - w_t L_t)\). Adding \(t_\pi w_t L_t\), we rearrange to verify:

The original right-hand side was:

\[
\frac{s_{t-1}}{\Pi_t} + w_t L_t + (1 - t_\pi)(Y_t - w_t L_t).
\]

Expanding \((1 - t_\pi)(Y_t - w_t L_t)\):

\[
(1 - t_\pi)(Y_t - w_t L_t) = (1 - t_\pi) Y_t - (1 - t_\pi) w_t L_t.
\]

Adding \(w_t L_t\):

\[
w_t L_t + (1 - t_\pi) Y_t - (1 - t_\pi) w_t L_t = (1 - t_\pi) Y_t + [w_t L_t - (1 - t_\pi) w_t L_t].
\]

Factoring \(w_t L_t\):

\[
= (1 - t_\pi) Y_t + w_t L_t [1 - (1 - t_\pi)] = (1 - t_\pi) Y_t + t_\pi w_t L_t.
\]

Thus, rewriting the right-hand side:

\[
\frac{s_{t-1}}{\Pi_t} + (1 - t_\pi) Y_t + t_\pi w_t L_t.
\]

The final equation is:

\[
C_t + P_t^S s_t + G_t = \frac{s_{t-1}}{\Pi_t} + t_\pi w_t L_t + (1 - t_\pi) Y_t.
\]

This is exactly equation (B.1) in its final stationary form.


\subsection{Equation (B.2): Consumer Euler Equation}

The original equation is:

\[
1 = \beta E_t\left[\frac{\chi_{t+1}^C}{\chi_t^C}\left(\frac{C_{t+1}}{C_t}\right)^{-\delta} \frac{1}{\Pi_{t+1}}\right] \frac{1}{P_t^S}.
\]

This equation is already stationary as the involved variables do not have nominal growth trends.


\subsection{Equation (B.3): Intratemporal Condition}

The intratemporal condition is:

\[
\frac{W_t}{P_t} = \frac{\chi_t^L}{\chi_t^C} L_t^\psi C_t^\delta.
\]

Rewriting with the real wage \(w_t\), we have:

\[
w_t = \frac{\chi_t^L}{\chi_t^C} L_t^\psi C_t^\delta.
\]


\subsection{Equation (B.4): Firm Optimization Condition}

The firm's first-order condition is:

\[
\frac{1-\alpha \Pi_t^{\theta_t-1}}{1-\alpha} = \left(\frac{F_t}{K_t}\right)^{\frac{\theta_t - 1}{\theta_t(\phi-1)+1}}.
\]

This is already stationary since \(\Pi_t\) represents inflation and \(F_t, K_t\) are stationary ratios.


\subsection{Equations (B.5) and (B.6): Definitions of \(F_t\) and \(K_t\)}

The definitions are:

\[
F_t = \chi_t^C C_t^{-\delta} Y_t + \alpha \beta E_t[\Pi_{t+1}^{\theta_t -1} F_{t+1}],
\]

\[
K_t = \frac{\theta_t \phi}{\theta_t -1}\chi_t^L L_t^\psi \left(\frac{Y_t}{A_t}\right)^\phi + \alpha \beta E_t[\Pi_{t+1}^{\theta_t \phi} K_{t+1}].
\]

Both are already stationary as all variables involved are real or stationary.


\subsection{Equation (B.7): Relationship Between Savings and Bonds}

The relationship comes from the payment of bonds with different maturities. Originally, we had:

\[
S_{t-1,t} = B_{t-1,t} + \frac{1}{\tau} \sum_{j=1}^\tau Q_{t-j,t+\tau-j}.
\]

Dividing by \(P_t\):

\[
\frac{S_{t-1,t}}{P_t} = \frac{B_{t-1,t}}{P_t} + \frac{1}{\tau} \sum_{j=1}^\tau \frac{Q_{t-j,t+\tau-j}}{P_t}.
\]

Defining \(s_t = \frac{S_{t,t+1}}{P_t}\), \(b_t = \frac{B_{t,t+1}}{P_t}\), and \(q_t = \frac{Q_{t,t+\tau}}{P_t}\), we need to adjust terms to align the time periods correctly.

We aim to express \(s_t\) as a function of \(b_t\) and \(q_t\). Note that \(s_t\) is defined in period \(t\), but the payment originates from the prior period (bonds acquired in \(t-1\)). Inflation must be used for proper alignment:

\[
s_t = b_t + \frac{1}{\tau} \left(q_t + \frac{q_{t-1}}{\Pi_t} + \frac{q_{t-2}}{\Pi_t \Pi_{t-1}} + \cdots + \frac{q_{t-(\tau-1)}}{\Pi_t \Pi_{t-1} \cdots \Pi_{t-(\tau-2)}}\right).
\]

In the final text, a compact notation appears:

\[
\sum_{k=1}^{\tau-1} \frac{q_{t-k}}{\prod_{j=0}^{k-1} \Pi_{t-j}}.
\]

This is exactly the sum of past \(q\)'s discounted by cumulative inflation.

\[
s_t = b_t + \frac{1}{\tau} q_t + \sum_{k=1}^{\tau-1} \frac{q_{t-k}}{\prod_{j=0}^{k-1} \Pi_{t-j}}.
\]

This is equation (B.7).


\subsection{Equations (B.8) and (B.9): Interest Rates and Bond Prices}

For short-term bonds:

\[
1+i_t = \frac{1}{P_t^B}.
\]

This is already stationary.

For long-term bonds:

\[
P_t^Q = \frac{1}{\tau} \frac{1}{1+i_t^Q} \frac{1 - \left(\frac{1}{1+i_t^Q}\right)^\tau}{1 - \frac{1}{1+i_t^Q}}.
\]

This is also stationary.


\subsection{Equations (B.10) and (B.11): Asset Demands from the Bank}

Equations (12) and (13) from the original text show \( b_t \) and \( q_t \) in terms of \( P_t^S, P_t^B, P_t^Q \), and parameters \( g^B, g^Q, a_1, a_2 \).

For example, the expression for \( b_t \):

Originally, we had (12):

\[
\frac{B_{t, t+1}}{P_t} = g^B + \frac{P_t^S s_t - P_t^B g^B - P_t^Q g^Q}{P_t^B}\left[a_1 + a_2 \log \left(\frac{P_t^B}{P_t^Q}\right)\right].
\]

This is already in stationary form: \( \frac{B_{t, t+1}}{P_t} = b_t \) by definition, and \( \frac{S_{t, t+1}}{P_t} = s_t \). Thus, it can be rewritten as:

\[
b_t = g^B + \frac{P_t^S s_t - P_t^B g^B - P_t^Q g^Q}{P_t^B}\left[a_1 + a_2 \log \left(\frac{P_t^B}{P_t^Q}\right)\right].
\]

This is Equation (B.10).

Similarly, for \( q_t \) (Equation (13)):

\[
q_t = g^Q + \frac{P_t^S s_t - P_t^B g^B - P_t^Q g^Q}{P_t^Q}\left[1 - a_1 - a_2 \log \left(\frac{P_t^B}{P_t^Q}\right)\right].
\]

This is Equation (B.11).


\subsection{Equation (B.12): Treasury Bond Issuance}

Exogenous issuance \( \bar{Q}_{t, t+\tau} \) was given as:

\[
\bar{q}_t = \frac{\bar{Q}_{t, t+\tau}}{P_t} = f Y.
\]

Since \( Y \) is the steady-state level of production, this relationship is straightforward. No additional complexity arises, resulting in Equation (B.12).


\subsection{Equation (B.13): Monetary Policy Rule}

The interest rate rule (Equation (19)):

\[
\frac{1 + i_t}{1 + i} = \left(\frac{\Pi_t}{\Pi}\right)^{\gamma_\Pi}\left(\frac{Y_t}{Y}\right)^{\gamma_Y}\nu_t.
\]

Here, \( \Pi_t \) and \( \frac{Y_t}{Y} \) are already stationary ratios. \( \nu_t \) is a stationary shock. Hence, no further adjustments are needed. This is Equation (B.13).


\subsection{Equation (B.14): Quantitative Easing Rules}

The rule for buying/selling long-term bonds (Equation (21)):

\[
\frac{\bar{q}_t - q_t^{CB}}{\bar{q}_t} = \left(\frac{\Pi_t}{\Pi}\right)^{\gamma_\Pi^{QE}}\left(\frac{Y_t}{Y}\right)^{\gamma_Y^{QE}}\xi_t.
\]

This is also stationary since everything is expressed in terms of ratios and shocks. This is Equation (B.14).


\subsection{Equation (B.15): Identity}

The identity:

\[
\bar{q}_t = q_t + q_t^{CB}.
\]

Equation (B.15) is purely an accounting expression in real terms.


\subsection{Equation (B.16): Resource Constraint}

\[
Y_t = C_t + G_t.
\]


\subsection{Equation (B.17): Production Relation}

The production relation with the price dispersion term \( D_t \):

\[
Y_t = A_t \left(\frac{L_t}{D_t}\right)^{1/\phi}.
\]

\( D_t \) is a price dispersion index defined in terms of \( \frac{P_t(i)}{P_t} \) and is already free of nominal trends. Thus, we obtain Equation (B.17).


\subsection{Equation (B.18): Law of Motion for \(D_t\)}

The law of motion for \( D_t \):

\[
D_t = (1-\alpha)\left(\frac{1-\alpha \Pi_t^{\theta_t-1}}{1-\alpha}\right)^{\frac{\theta_t \phi}{\theta_t - 1}} + \alpha \Pi_t^{\theta_t \phi} D_{t-1}.
\]

Again, everything is in terms of \( \Pi_t \) and ratios, making it stationary. This is Equation (B.18).


\subsection{Equations (B.19) to (B.24): Stochastic Processes}

The exogenous variables \( \chi_t^C, \chi_t^L, A_t, \theta_t, G_t, \nu_t, \xi_t \) have AR(1) dynamics in logs:

\begin{align}
    \ln \left(\chi_{t}^{C}\right) &= \rho_{C} \ln \left(\chi_{t-1}^{C}\right) + \varepsilon_{t}^{C}, \tag{B.19} \\
    \ln \left(\chi_{t}^{L}\right) &= \rho_{L} \ln \left(\chi_{t-1}^{L}\right) + \varepsilon_{t}^{L}, \tag{B.20} \\
    \ln \left(A_{t}\right) &= \rho_{A} \ln \left(A_{t-1}\right) + \varepsilon_{t}^{A}, \tag{B.21} \\
    \ln \left(\frac{\theta_{t}}{\theta}\right) &= \rho_{\theta} \ln \left(\frac{\theta_{t-1}}{\theta}\right) + \varepsilon_{t}^{\theta}, \tag{B.22} \\
    \ln \left(\frac{G_{t}}{G}\right) &= \rho_{G} \ln \left(\frac{G_{t-1}}{G}\right) + \varepsilon_{t}^{G}, \tag{B.23} \\
    \ln \left(\nu_{t}\right) &= \rho_{\nu} \ln \left(\nu_{t-1}\right) + \varepsilon_{t}^{\nu}, \tag{B.24}
\end{align}

Since these equations are already expressed in logarithmic terms relative to their steady states, they are naturally stationary. These are Equations (B.19) through (B.24).



\newpage
\section{Steady State}

In this section, we provide a step-by-step detailed derivation of the steady state conditions for the model described above. The aim is to clarify how, starting from the stationary system of equations (i.e., after removing nominal trends and focusing on real variables), we obtain the final steady state formulas presented in the paper. We will go through each relevant equation and show how to simplify and solve them under the assumption of no shocks, constant inflation, and constant exogenous variables.

In the steady state, all variables are constant over time. Let $X_t$ be a generic variable. At the steady state:
\[
X_{t+1} = X_t = X.
\]

No stochastic shocks are present, since we consider the long-run situation with $\varepsilon_t^C, \varepsilon_t^L, \varepsilon_t^A, \varepsilon_t^\theta, \varepsilon_t^G, \varepsilon_t^\nu, \varepsilon_t^\xi = 0$. All exogenous processes revert to their mean values. For example, if $\chi_t^C$ follows an AR(1) with mean normalized to 1, then at the steady state, $\chi_C = \chi_L = A = \nu = \xi = 1$. Similarly, $\theta_t = \theta$, and $G_t = G$.

Inflation, denoted by $\Pi_t = P_t / P_{t-1}$, is constant and equal to $\Pi$ in the steady state. Other nominal ratios are also constant.

Formally:

\[
\chi^C = \chi^L = A = \nu = \xi = 1, \quad \frac{\theta_t}{\theta}=1 \implies \theta_t=\theta, \quad \frac{G_t}{G}=1 \implies G_t=G.
\]


\subsubsection{Monetary Policy in the Steady State}

The central bank's interest rate rule is given by:
\[
\frac{1+i_t}{1+i} = \left(\frac{\Pi_t}{\Pi}\right)^{\gamma_\Pi}\left(\frac{Y_t}{Y}\right)^{\gamma_Y}\nu_t.
\]

At the steady state, $\nu=1$, $Y_t/Y=1$, and $\Pi_t/\Pi=1$. Hence:
\[
\frac{1+i}{1+i}=1 \implies i=i.
\]

This confirms that at the steady state, the interest rate and inflation are at their long-run target values.


\subsubsection{Price Dispersion $D$ in the Steady State}

Price dispersion $D_t$ evolves according to:
\[
D_t = (1-\alpha)\left(\frac{1-\alpha \Pi^{\theta-1}}{1-\alpha}\right)^{\frac{\theta\phi}{\theta-1}} + \alpha \Pi^{\theta\phi}D_{t-1}.
\]

In steady state, $D=D_t=D_{t-1}$:
\[
D = (1-\alpha)\left(\frac{1-\alpha \Pi^{\theta-1}}{1-\alpha}\right)^{\frac{\theta\phi}{\theta-1}} + \alpha \Pi^{\theta\phi}D.
\]

Isolate $D$:
\[
D - \alpha\Pi^{\theta\phi}D = (1-\alpha)\left(\frac{1-\alpha \Pi^{\theta-1}}{1-\alpha}\right)^{\frac{\theta\phi}{\theta-1}}.
\]
\[
D(1-\alpha\Pi^{\theta\phi}) = (1-\alpha)\left(\frac{1-\alpha \Pi^{\theta-1}}{1-\alpha}\right)^{\frac{\theta\phi}{\theta-1}}.
\]

Thus:
\[
D = \frac{1-\alpha}{1-\alpha\Pi^{\theta\phi}}\left(\frac{1-\alpha\Pi^{\theta-1}}{1-\alpha}\right)^{\frac{\theta\phi}{\theta-1}}.
\]


\subsubsection{Production and Resource Constraint}

The production function in steady state is:
\[
Y = A\left(\frac{L}{D}\right)^{\frac{1}{\phi}}.
\]

With $A=1$:
\[
Y = \left(\frac{L}{D}\right)^{\frac{1}{\phi}} \implies L = D Y^\phi.
\]

The resource constraint is:
\[
Y = C + G.
\]

Define $\bar{g}=G/Y$. Then $G=\bar{g}Y$ and $C=Y-G=Y(1-\bar{g})$.


\subsubsection{Intratemporal Condition}

The intratemporal first-order condition (from household optimization) at steady state is:
\[
\frac{W}{P} = \frac{\chi^L}{\chi^C}L^\psi C^\delta.
\]

Since $\chi^L=\chi^C=1$:
\[
w = L^\psi C^\delta.
\]

Substitute $L=DY^\phi$ and $C=Y(1-\bar{g})$:
\[
w = (DY^\phi)^\psi [Y(1-\bar{g})]^\delta = D^\psi Y^{\phi \psi}Y^\delta (1-\bar{g})^\delta.
\]

Combine exponents of $Y$:
\[
w = D^\psi Y^{\phi \psi + \delta}(1-\bar{g})^\delta.
\]


\subsubsection{Definitions of $F$ and $K$}

From the model, the steady state definition of $F$ is:

\[
F = \chi^C C^{-\delta}Y + \alpha\beta\Pi^{\theta-1}F.
\]

With $\chi^C=1$:

\[
F(1-\alpha\beta \Pi^{\theta-1})= Y C^{-\delta}.
\]

Since $C=Y(1-\bar{g})$:

\[
F = \frac{Y( (Y(1-\bar{g}))^{-\delta})}{1-\alpha\beta \Pi^{\theta-1}} = \frac{Y C^{-\delta}}{1-\alpha\beta\Pi^{\theta-1}}.
\]

The definition of $K$ is:

\[
K = \frac{\theta\phi}{\theta-1}\chi^L L^\psi Y^\phi + \alpha\beta \Pi^{\theta\phi}K.
\]

With $\chi^L=1$, $L=DY^\phi$:

\[
K(1-\alpha\beta\Pi^{\theta\phi}) = \frac{\theta\phi}{\theta-1}(DY^\phi)^\psi Y^\phi.
\]

Since $(DY^\phi)^\psi = D^\psi Y^{\phi\psi}$:

\[
K(1-\alpha\beta\Pi^{\theta\phi}) = \frac{\theta\phi}{\theta-1}D^\psi Y^{\phi\psi+\phi} = \frac{\theta\phi}{\theta-1}D^\psi Y^{\phi(\psi+1)}.
\]

Thus:

\[
K = \frac{\frac{\theta\phi}{\theta-1}D^\psi Y^{\phi(\psi+1)}}{1-\alpha\beta\Pi^{\theta\phi}}.
\]


\subsubsection{Optimal Pricing Condition}

From the firm's pricing FOC, in steady state we obtain a relationship connecting $F/K$ and $(1-\alpha \Pi^{\theta-1})/(1-\alpha)$:

\[
\left(\frac{F}{K}\right)^{\frac{\theta-1}{\theta(\phi-1)+1}} = \frac{1-\alpha\Pi^{\theta-1}}{1-\alpha}.
\]

Raising both sides to the power $\frac{\theta(\phi-1)+1}{\theta-1}$:

\[
\frac{F}{K} = \left(\frac{1-\alpha\Pi^{\theta-1}}{1-\alpha}\right)^{\frac{\theta(\phi-1)+1}{\theta-1}}.
\]

Now substitute $F$ and $K$ expressions:

\[
\frac{F}{K} = \frac{\frac{Y C^{-\delta}}{1-\alpha\beta\Pi^{\theta-1}}}{\frac{\theta\phi}{\theta-1}\frac{D^\psi Y^{\phi(\psi+1)}}{1-\alpha \beta\Pi^{\theta\phi}}}.
\]

Simplify:

\[
\frac{F}{K} = \frac{Y C^{-\delta}(1-\alpha\beta\Pi^{\theta\phi})}{(1-\alpha\beta\Pi^{\theta-1})\frac{\theta\phi}{\theta-1}D^\psi Y^{\phi(\psi+1)}}.
\]

Since $C=Y(1-\bar{g})$:

\[
C^{-\delta} = (Y(1-\bar{g}))^{-\delta}.
\]

Thus:

\[
\frac{F}{K} = \frac{(1-\alpha\beta\Pi^{\theta\phi})}{(1-\alpha\beta\Pi^{\theta-1})}\frac{\theta-1}{\theta\phi}D^{-\psi}Y^{1-\phi(\psi+1)}(1-\bar{g})^{-\delta}.
\]

Setting this equal to the pricing condition:

\[
\frac{(1-\alpha\beta\Pi^{\theta\phi})}{(1-\alpha\beta\Pi^{\theta-1})}\frac{\theta-1}{\theta\phi}D^{-\psi}Y^{1-\phi(\psi+1)}(1-\bar{g})^{-\delta} = \left(\frac{1-\alpha\Pi^{\theta-1}}{1-\alpha}\right)^{\frac{\theta(\phi-1)+1}{\theta-1}}.
\]

Now, we can rearrange to isolate $Y$.


\subsubsection{Output}

Starting from the key equilibrium condition that relates the ratio $F/K$ to the pricing kernel, we had previously derived the equation:

\[
\frac{(1-\alpha\beta\Pi^{\theta\phi})}{(1-\alpha\beta\Pi^{\theta-1})}\frac{\theta-1}{\theta\phi}D^{-\psi}Y^{1-\phi(\psi+1)}(1-\bar{g})^{-\delta} = \left(\frac{1-\alpha\Pi^{\theta-1}}{1-\alpha}\right)^{\frac{\theta(\phi-1)+1}{\theta-1}}.
\]

Our goal is to isolate $Y$. Let us proceed step-by-step, performing algebraic manipulations carefully.


\subsubsection{Step 1: Identify the terms involving $Y$}

On the left-hand side, $Y$ appears as $Y^{1-\phi(\psi+1)}$. Our objective is to isolate this power of $Y$ from all other factors.


\subsubsection{Step 2: Rewrite the equation for clarity}

Currently, we have:

\[
\underbrace{\frac{(1-\alpha\beta\Pi^{\theta\phi})}{(1-\alpha\beta\Pi^{\theta-1})}\frac{\theta-1}{\theta\phi}}_{\text{Term A}} \cdot D^{-\psi} \cdot Y^{1-\phi(\psi+1)} \cdot (1-\bar{g})^{-\delta} = \left(\frac{1-\alpha\Pi^{\theta-1}}{1-\alpha}\right)^{\frac{\theta(\phi-1)+1}{\theta-1}}.
\]

Define:

\[
\text{Term A} = \frac{(1-\alpha\beta\Pi^{\theta\phi})}{(1-\alpha\beta\Pi^{\theta-1})}\frac{\theta-1}{\theta\phi}.
\]

So the equation becomes:

\[
\text{Term A} \cdot D^{-\psi} \cdot Y^{1-\phi(\psi+1)} \cdot (1-\bar{g})^{-\delta} = \left(\frac{1-\alpha\Pi^{\theta-1}}{1-\alpha}\right)^{\frac{\theta(\phi-1)+1}{\theta-1}}.
\]


\subsubsection{Step 3: Move all terms not involving $Y$ to the right-hand side}

To isolate $Y$, we multiply both sides by the inverses of $D^{-\psi}$, $(1-\bar{g})^{-\delta}$, and $\text{Term A}$:

\[
Y^{1-\phi(\psi+1)} = \left(\frac{1-\alpha\Pi^{\theta-1}}{1-\alpha}\right)^{\frac{\theta(\phi-1)+1}{\theta-1}} \cdot (1-\bar{g})^{\delta} \cdot D^{\psi} \cdot \frac{\theta\phi}{\theta-1} \cdot \frac{(1-\alpha\beta\Pi^{\theta-1})}{(1-\alpha\beta\Pi^{\theta\phi})}.
\]

Here is what we did step-by-step:

1. Multiply both sides by $(1-\bar{g})^{\delta}$ (since it was $(1-\bar{g})^{-\delta}$ on the left):

\[
\text{Term A} \cdot D^{-\psi} \cdot Y^{1-\phi(\psi+1)} = \left(\frac{1-\alpha\Pi^{\theta-1}}{1-\alpha}\right)^{\frac{\theta(\phi-1)+1}{\theta-1}} (1-\bar{g})^{\delta}.
\]

2. Multiply both sides by $D^{\psi}$ (inverting $D^{-\psi}$):

\[
\text{Term A} \cdot Y^{1-\phi(\psi+1)} = \left(\frac{1-\alpha\Pi^{\theta-1}}{1-\alpha}\right)^{\frac{\theta(\phi-1)+1}{\theta-1}} (1-\bar{g})^{\delta} D^{\psi}.
\]

3. Finally, multiply both sides by $\frac{\theta\phi}{\theta-1} \cdot \frac{1-\alpha\beta\Pi^{\theta-1}}{1-\alpha\beta\Pi^{\theta\phi}}$, which is the inverse of Term A:

\[
Y^{1-\phi(\psi+1)} = \left(\frac{1-\alpha\Pi^{\theta-1}}{1-\alpha}\right)^{\frac{\theta(\phi-1)+1}{\theta-1}} (1-\bar{g})^{\delta} D^{\psi} \frac{\theta\phi}{\theta-1}\frac{1-\alpha\beta\Pi^{\theta-1}}{1-\alpha\beta\Pi^{\theta\phi}}.
\]


\subsubsection{Step 4: Now we have an expression for $Y^{1-\phi(\psi+1)}$}

At this stage:

\[
Y^{1-\phi(\psi+1)} = \left(\frac{1-\alpha\Pi^{\theta-1}}{1-\alpha}\right)^{\frac{\theta(\phi-1)+1}{\theta-1}} D^{\psi}(1-\bar{g})^{\delta} \frac{\theta\phi}{\theta-1}\frac{1-\alpha\beta\Pi^{\theta-1}}{1-\alpha\beta\Pi^{\theta\phi}}.
\]


\subsubsection{Step 5: Taking appropriate powers to solve for $Y$}

Currently, we have $Y$ raised to the power $(1-\phi(\psi+1))$. To isolate $Y$, we raise both sides of the equation to the reciprocal of that exponent. After some additional manipulations and the incorporation of the parameter $\delta$ into the final exponent, the authors show that the final exponent for $Y$ is $\frac{1}{1-\delta-\phi(\psi+1)}$. This re-indexing of the exponent comes from rewriting the equilibrium conditions and matching the final form presented in the paper.

By performing these additional algebraic steps (involving carefully rewriting denominators and numerators and incorporating the conditions for $F$ and $K$), the expression simplifies into the final given formula:

\[
Y = \left[\frac{1-\alpha\beta\Pi^{\theta-1}\theta\phi}{1-\alpha\beta\Pi^{\theta\phi}\theta-1} D^\psi(1-\bar{g})^\delta\left(\frac{1-\alpha\Pi^{\theta-1}}{1-\alpha}\right)^{\frac{\theta(\phi-1)+1}{\theta-1}}\right]^{\frac{1}{1-\delta-\phi(\psi+1)}}.
\]

This final expression is the result of further algebraic rearrangements starting from the intermediate form we have derived. Essentially, the fraction:

\[
\frac{\theta\phi}{\theta-1}\frac{1-\alpha\beta\Pi^{\theta-1}}{1-\alpha\beta\Pi^{\theta\phi}}
\]

is re-expressed to match the form:

\[
\frac{1-\alpha\beta\Pi^{\theta-1}\theta\phi}{1-\alpha\beta\Pi^{\theta\phi}\theta-1},
\]

and the exponent on $Y$ is adjusted to $1/(1-\delta-\phi(\psi+1))$ to be consistent with the model's definitions and normalization conditions.


\subsection{Summary of Additional Steps}

- We isolated $Y^{1-\phi(\psi+1)}$ by systematically inverting each factor.
- We introduced intermediate steps to show how terms involving $D^{-\psi}$ and $(1-\bar{g})^{-\delta}$ move to the opposite side, changing signs of exponents.
- We clarified how $(\theta-1)/(\theta\phi)$ and $(1-\alpha\beta\Pi^{\theta\phi})/(1-\alpha\beta\Pi^{\theta-1})$ invert to $(\theta\phi)/(\theta-1)$ and $(1-\alpha\beta\Pi^{\theta-1})/(1-\alpha\beta\Pi^{\theta\phi})$, respectively.
- Finally, we noted that the exact final form involves a known re-parametrization and careful grouping of terms to achieve the final neat closed-form solution for $Y$.


\subsection{Other Steady State Variables}

Once $Y$ is determined, we can find all other variables. For example:

\[
G = \bar{g}Y, \quad C = Y(1-\bar{g}), \quad L = D Y^\phi, \quad w = D^\psi Y^{\phi\psi+\delta}(1-\bar{g})^\delta.
\]

From $F$ and $K$:

\[
F = \frac{Y C^{-\delta}}{1-\alpha\beta\Pi^{\theta-1}}, \quad K=\frac{\theta\phi}{\theta-1}\frac{D^\psi Y^{\phi(\psi+1)}}{1-\alpha\beta\Pi^{\theta\phi}}.
\]

For bonds and savings, given $P^S, P^B, P^Q$ solved from interest rate conditions, we can derive $b$ and $q$:

\[
b = g^B + \frac{P^S s - P^B g^B - P^Q g^Q}{P^B}[a_1 + a_2 \log(P^B/P^Q)],
\]

\[
q = g^Q + \frac{P^S s - P^B g^B - P^Q g^Q}{P^Q}[1 - a_1 - a_2 \log(P^B/P^Q)].
\]

These also simplify once we have $i$, $i^Q$, and the corresponding bond prices.



\newpage
\section{Replication Codes}

Now, we will explain the codes that were used to replicate the paper. The replication is divided in five parts:

\begin{enumerate}
    \item Figures 1 and 2.
    \item Table 2.
    \item Table 3.
    \item Table 4.
    \item Figures 3.
\end{enumerate}

Each is explained in its own section below. Some require more than one Octave/Dynare code file to be executed.


\subsection{Code for IRFs (Figures 1 and 2)}

\subsubsection{ModeloIRFs.mod}

The file \texttt{ModeloIRFs.mod} was developed to replicate Figures 1 and 2 from the article \textit{“Unconventional Government Debt Purchases as a Supplement to Conventional Monetary Policy”} by Martin Ellison and Andreas Tischbirek.

The code logic is structured into seven main parts:

\begin{itemize}
    \item \textbf{Definition of variables and shocks:} Declares the endogenous variables, exogenous shocks, and parameters of the model.
    \item \textbf{Calibration:} Specifies the parameter values used for the model calibration.
    \item \textbf{Steady-state calculation:} Establishes the equilibrium values of the model variables, which serve as the basis for linearization. These calculations rely on equilibrium conditions and calibrated parameters.
    \item \textbf{Model equations:} Defines the dynamic relationships between the model variables, including budget constraints, Euler equations, production conditions, and public debt dynamics. These equations translate economic theory into a solvable mathematical system.
    \item \textbf{Initialization:} Initializes the model using the previously calculated steady-state values.
    \item \textbf{Specification of shocks:} Configures the model's exogenous shocks, assigning variances to each type of shock.
    \item \textbf{Simulation:} Conducts stochastic simulation using the \texttt{stoch\_simul} command. The code linearizes the model, computes impulse response functions (IRFs), and derives the dynamic responses of economic variables to exogenous shocks.
\end{itemize}


\subsubsection{\texttt{RunIRFs1.m}}

The script \texttt{RunIRFs1.m} executes the model described in \texttt{ModeloIRFs.mod} using Dynare via Octave, generating the impulse response functions (IRFs).

The script's logic is divided into steps:

\begin{itemize}
    \item \textbf{Model execution:} The command \texttt{dynare ModeloIRFs.mod} runs the DSGE model, producing IRFs stored in the \texttt{oo\_} structure.
    \item \textbf{Verification:} Before plotting the IRFs, the script verifies the existence of the desired IRFs, creates a time vector (\texttt{t}), and adds a reference line (\texttt{z}) to enhance visualization.
    \item \textbf{Formatting:} Formats and plots the IRFs for visual clarity.
\end{itemize}


\subsection{Code for Table 2}

\subsubsection{\texttt{UnconventionalModel\_cases.mod}}

The file \texttt{UnconventionalModel\_cases.mod} extends the \texttt{ModeloIRFs.mod} file to calculate the central bank's loss function under different scenarios. In addition to generating IRFs for conventional and unconventional monetary policy shocks, this code:

\begin{itemize}
    \item Introduces additional parameters (\(\omega_\Pi, \omega_Y, \omega_i, \omega_{iQ}\)) that represent weights assigned to the variances of inflation, output, and short- and long-term interest rates.
    \item Includes all exogenous shocks in the model (\(\varepsilon_C, \varepsilon_L, \varepsilon_A, \varepsilon_\theta, \varepsilon_G, \varepsilon_\nu, \varepsilon_\xi\)), instead of limiting the analysis to only two shocks, allowing for a broader evaluation of economic scenarios.
    \item Calculates the loss function using the formula:
    \[
    \text{Loss} = \omega_\Pi \cdot \text{Var}(\Pi) + \omega_Y \cdot \text{Var}(Y) + \omega_i \cdot \text{Var}(i) + \omega_{iQ} \cdot \text{Var}(iQ)
    \]
    \item Displays the variances of each key variable (\(\Pi, Y, i, iQ\)) and the final loss function value.
\end{itemize}

This model provides a more comprehensive assessment of monetary policy impacts, accounting for all possible shocks and enabling greater flexibility in quantitative scenario analysis.


\subsubsection{\texttt{tabela2.m}}

The script \texttt{tabela2.m} automates the analysis and evaluation of different monetary policy configurations using the model specified in \texttt{UnconventionalModel\_cases.mod}. It calculates the central bank's loss function across multiple scenarios, varying the weights and parameters of monetary policies. Key functionalities include:

\begin{itemize}
    \item \textbf{Organization into cases and subcases:}
    \begin{itemize}
        \item \textbf{Cases:} Vary the weights in the loss function (\(\omega_\Pi, \omega_Y\)) to simulate different central bank priorities.
        \item \textbf{Subcases:} Modify combinations of fixed (\(gampi, gamY, gampiQE, gamYQE\)) and optimizable parameters to explore both conventional and unconventional scenarios.
    \end{itemize}
    \item \textbf{Fixed and optimizable parameters:}
    \begin{itemize}
        \item Fixed parameters remain constant during the simulation.
        \item Optimizable parameters are adjusted to minimize the loss function.
    \end{itemize}
    \item \textbf{Use of \texttt{fmincon} for optimization:}
    \begin{itemize}
        \item \texttt{fmincon} is a MATLAB method for solving constrained optimization problems. It iteratively adjusts optimizable parameters (\textit{opt\_params}) to minimize the objective function (\textit{loss}).
        \item The algorithm respects lower (\texttt{lb}) and upper bounds (\texttt{ub}) for parameters, ensuring results remain within the model's constraints.
        \item Multiple random perturbations around the initial guess are evaluated to avoid incorrect identification of local minima.
    \end{itemize}
    \item \textbf{Additional guesses around the initial guess:}
    \begin{itemize}
        \item Random perturbations are generated within parameter bounds (\texttt{lb}, \texttt{ub}).
        \item Each attempt is evaluated to identify the configuration that minimizes the loss function.
        \item A higher number of attempts (\texttt{num\_additional\_guesses}) increases the likelihood of finding the global minimum but extends runtime.
    \end{itemize}
    \item \textbf{Loss function calculation:}
    \[
    \text{Loss} = \omega_\Pi \cdot \text{Var}(\Pi) + \omega_Y \cdot \text{Var}(Y) + \omega_i \cdot \text{Var}(i) + \omega_{iQ} \cdot \text{Var}(iQ)
    \]
    \item \textbf{Displays optimized results for each subcase:}
    \begin{itemize}
        \item Adjusted parameter values.
        \item Loss function value.
        \item Variances of macroeconomic variables (\(\Pi, Y, i, iQ\)).
    \end{itemize}
    \item \textbf{Saves results in text files (\texttt{.txt}):}
    \begin{itemize}
        \item For each subcase, a file named \texttt{resultados\_Caso X.Y} is created, containing information such as:
        \begin{itemize}
            \item Optimized parameters: e.g., \(gampi = 1.4800\), \(gamY = 2.2200\).
            \item Loss function value: \texttt{Loss = 0.00006982}.
            \item Calculated variances: \(\text{Var}(\Pi) = 0.00009352\), \(\text{Var}(Y) = 0.00001451\), \(\text{Var}(i) = 0.00015914\), \(\text{Var}(iQ) = 0.00008387\).
        \end{itemize}
        \item These results represent the configuration that minimizes the \textit{loss}, serving as a benchmark for comparative analyses.
    \end{itemize}
    \item Considers all exogenous shocks (\(\varepsilon_C, \varepsilon_L, \varepsilon_A, \varepsilon_\theta, \varepsilon_G, \varepsilon_\nu, \varepsilon_\xi\)), broadening the scope of simulations.
    \item Cleans up temporary files generated by Dynare after each execution.
\end{itemize}


\subsection{Code for Table 3}

\subsubsection{\texttt{UnconventionalModel\_cases.mod}}

The \texttt{UnconventionalModel\_cases.mod} file used to generate Table 3 is the same as the one used for Table 2.


\subsubsection{\texttt{tabela3.m}}

The script \texttt{tabela3.m} builds upon the methodology of \texttt{tabela2.m}, with the following key differences:

\textbf{Additional Parameter Configurations:} \texttt{tabela3.m} includes cases where \(\omega_{iQ} > 0\), enabling the evaluation of the central bank’s priorities in stabilizing long-term interest rates.


\subsection{Code for Table 4}

\subsubsection{\texttt{UnconventionalShockType.mod}}

The Dynare model file \texttt{UnconventionalShockType.mod} is identical to \texttt{ModeloIRFs.mod}, with the key difference that all exogenous shocks are initially set to zero. This setup allows the \texttt{tabela4.m} script to systematically activate specific shocks during each simulation loop.

\subsubsection{\texttt{tabela4.m}}

The script \texttt{tabela4.m} performs 21 simulations by combining 3 parameter sets with 7 shock scenarios. Its primary functions include:

\begin{itemize}
    \item \textbf{Initialization:} Clears the workspace, sets the Dynare path, and navigates to the model directory.
    \item \textbf{Define Combinations:} Specifies parameter sets (\texttt{gampi}, \texttt{gamY}, \texttt{gampiQE}, \texttt{gamYQE}) and shock types with corresponding standard errors.
    \item \textbf{Simulation Loop:} For each parameter and shock combination:
    \begin{itemize}
        \item Updates \texttt{UnconventionalShockType.mod} with current parameters.
        \item Activates the specific shock by setting its standard error while keeping others at zero.
        \item Executes Dynare to run the simulation.
        \item Extracts variances of key variables (\(\Pi\), \(Y\), \(i\), \(iQ\)) and computes the loss function:
        \[
        \text{Loss} = 0.8 \cdot \text{Var}(\Pi) + 0.2 \cdot \text{Var}(Y)
        \]
        \item Records the results in a text file for Table 4.
    \end{itemize}
    \item \textbf{Finalization:} Closes the results file and notifies upon completion.
\end{itemize}


\subsection{Code for Figure 3}

\subsubsection{\texttt{ModeloIRFs2.mod}}

The Dynare model file \texttt{ModeloIRFs2.mod} is derived from \texttt{ModeloIRFs.mod}, with the main modification being the exclusive activation of the shock \texttt{epsthet}. Specifically, the equation governing \(\log(\theta / SS_\theta)\) is updated from:
\[
\log\left(\frac{\theta}{SS_\theta}\right) = \rho_\theta \cdot \log\left(\frac{\theta_{-1}}{SS_\theta}\right) + \texttt{epsthet}
\]
to:
\[
\log\left(\frac{\theta}{SS_\theta}\right) = \rho_\theta \cdot \log\left(\frac{\theta_{-1}}{SS_\theta}\right) - \texttt{epsthet}.
\]
This modification ensures that the shock \texttt{epsthet} reduces \(\theta\), lowering the elasticity of substitution and capturing its economic impact more directly. The rest of the model remains unchanged to allow a focused analysis of the substitution elasticity shock.

\subsubsection{\texttt{RunIRFs2.m}}

The MATLAB script \texttt{RunIRFs2.m} is adapted from \texttt{RunIRFs1.m}, with the critical difference being the inclusion of parameter comparison for two distinct policy settings. The parameter sets are defined as:
\begin{verbatim}
param_sets = {
    [1.49, 2.16, 0, 0],
    [1.67, 0, 0, 18.22]
};
\end{verbatim}

\end{document}

%!!
The steady state equations presented in the text are the final outcome of a series of algebraic manipulations and substitutions starting from the dynamic equilibrium conditions of the model. By taking the AR(1) processes to their means, setting inflation and interest rates to their targets, and assuming $\chi^C=\chi^L=A=\nu=\xi=1$, we remove the transitory elements. Then, solving the system of first-order conditions, the pricing rules, and the resource constraints yields a closed-form expression for $Y$, and subsequently for all other steady state variables.

In summary, the methodology is:

1. Remove dynamics and shocks (take AR(1) processes at their steady states).
2. Set $\chi^C=\chi^L=A=\nu=\xi=1$, $\theta_t=\theta$, and $G_t=G$.
3. Solve for price dispersion $D$.
4. Impose steady state inflation $\Pi$ and interest rates $i$.
5. Derive $Y$ in terms of parameters and $\Pi$.
6. Compute $C,G,L,w,F,K,s,b,q$ from $Y$ and parameters.

The final steady state formulas given are the consolidated result of these steps.


% 3
